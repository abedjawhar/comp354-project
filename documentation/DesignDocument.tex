\documentclass[12pt]{article}

\usepackage{color}
\usepackage{nth}
\usepackage{enumitem}
\usepackage{booktabs}
\usepackage{tabularx}
\usepackage{hyperref}
\usepackage[pdftex]{graphicx}
\pagestyle{empty}
\setcounter{secnumdepth}{2}
\usepackage{float}
\usepackage{multirow}

\pagestyle{empty}
\setcounter{secnumdepth}{2}

\topmargin=0cm
\oddsidemargin=0cm
\textheight=22.0cm
\textwidth=16cm
\parindent=0cm
\parskip=0.15cm
\topskip=0truecm
\raggedbottom
\abovedisplayskip=3mm
\belowdisplayskip=3mm
\abovedisplayshortskip=0mm
\belowdisplayshortskip=2mm
\normalbaselineskip=12pt
\normalbaselines

\begin{document}

\vspace*{0.5in}
\centerline{\bf\Large COMP 354}
\centerline{\bf\Large Design Document for myMoney}

\vspace*{0.5in}
\centerline{\bf\Large Team PA-PK}

\vspace*{0.5in}
\centerline{\today}

\vspace*{1.5in}
\begin{table}[htbp]
\caption{Team}
\begin{center}
\begin{tabular}{|r | c|}
\hline
Name & ID Number \\
\hline\hline
Anne-Laure Ehresmann & 27858906 \\
\hline
Marc-Antoine Dube & 40029307 \\
\hline
Kadeem Caines & 26343600 \\
\hline
Abdel Rahman Jawhar & 27192142 \\
\hline
Keith Dion & 40036340 \\
\hline
Hrachya Hakobyan & 40041555 \\
\hline
Andrew-Smith & 40034936 \\
\hline
Dongyu Chen & 27241909 \\
\hline
Yauheni Karaniuk & 40005680 \\
\hline
Renny Xu & 40005262\\
\hline
Wei Wang & 40041116 \\
\hline
\end{tabular}
\end{center}
\end{table}

\clearpage

\tableofcontents
\listoffigures
\listoftables

\clearpage


\section{Introduction and Purpose}

The goal of this document is to define the design for the desktop application myMoney. The majority of the design decisions have been taken     with the Requirements document in mind, one may thus want to look at this document first to have a clear picture of the problem in mind as well as the requirements demanded for the solution. This document presents an implementation of a possible solution to answer this problem. Its design is is outlined through an Architectural Design (AD), a Detailed design (DD) and Dynamic Design Scenarios (DDS) for the application. The AD focuses on high-level project decomposition, the DD describes the overarching system design (which includes the UML design, divided into multiple subsections), and the DDS displays how the subsystems interact with one another in order to produce system-level services. This document may thus be used to plan, coordinate, and guide the development of the software, estimate and allocate necessary resources for proper execution, and  then actually implement the software for the system. It seeks, above all, to serve as a precise and stable reference throughout development.

\section{Scope}

This document contains everything to do with the development decisions and design of the system, all of which are derived from the requirements, which are not described in this document. Also not included in here is any testing of the system, which verifies that the requirements are met. It is merely a blueprint for a system that should, in theory, successfully pass any tests that would be done in correspondence with the requirements.

\section{Architectural Design} \label{sec:arch}

The myMoney application uses the Model-View-Controller (MVC) architectural pattern at its core.

The \textit{view} is implemented through a JAVAFX front-end interface, which consists mostly of hard-coded tables structures or menus, which are then populated with table entries by the model, as requested by the view controllers. The view is interactive, and reports any events triggered by the user to the view controllers, which then pass the requests to the model controllers (services), which handle the modification of the model. Once updated, the model then passes its changes back to the view controllers, to update the information displayed to the user. A more in-depth view of the interaction between the user and the view can be seen in the dynamic models, in section \ref{Dynamic Models}.

The \textit{controllers} are the most complex part of the system. They are organised in a layered structure of services, which each handle a different sector of the application (session management, account services, user services, etc). All services perform their own validation, whether it is for implementing the business rules, or simply ensuring expected application behaviour (no null objects, caught exceptions...). Once this validation has been verified, the controllers passes the actually requested change to the model. Examples of such services, their intercommunication, and their validation, is explained more in-depth in section \ref{subsystem interface}.

The \textit{model}, which includes the back-end connection to the database, is composed of data access objects (DAOs), which are used to apply edits to the database, after the upper layers of the controllers have verified the validity of the calls. Our system actually employs two databases; The first, a local database holding user info, bank account info, and transaction info, and the second, a "remote" database (also local, but acts as if it were remote) used to simulate the bank servers. The only time the second database is actually accessed is when the user first adds a bank account. In this case, our view receives his input, passes it to the controls, which then makes requests for information to the "remote" database through the use of the services handling remote communication. These services receive, from the DAOs, a serialisation of the "remote" account, which it then translates into usable data for the local database. Once this has been successfully executed, the model triggers a view update, wherein the user can see his newly requested additions. All other events triggered by the user have no need of the remote database, and simply employ a series of communications between the controllers and the services handling local database. See section \ref{subsystem interface} for more details on these services and the databases.

\subsection{Architectural Diagram}

\begin{figure}[H]
\includegraphics[width=\linewidth]{package-diagram.png}
\end{figure}

We herewith present the architectural diagram of the design presented above, and provide an explanation of each package (but not for the subsystems within each package). A succint and precise description of each subsystem is available in the next section.

Notice first that MVC's style forces a clear separation of concerns, and thus emphasises a great amount of intercommunication between each section. To embrace this, using interfaces for each component presented above will clearly separate the implementation from the structure of the module, which eases parallel addition, modification, or testing of the system.

The com.github.comp354project.service package is the main subsystem of our project. As mentioned above, it is organised in a layered manner, wherein each layer handles its own services, and use the services of the layer below it within worrying about that layer's implementation. Data validation and processing is offered by each service: The account service, for example, validate calls to add or delete bank accounts, edit a transaction's category, or query for specific accounts. It does this by querying the database using an account DAO, and ensuring data integrity and validity (with regards to the business rules). It does not, however, worry about user authorisation, and simply assumes the layer above it (The user service) will have handled it. The com.github.comp354project.viewController calls services within this package to update the view and the model.

The com.github.comp354project.viewController package contains a number of controllers for each different view. They are the ones handling the requests from the users, which they pass to the services in the package described above. They are also the ones who pass any errors from the services to the views, mostly to be used for testing and alerting the user of any problems that might have occured.

The com.github.comp354project.service package.account.remote package is a subsystem to our services which mocks an API call to remote servers of banks or credit card companies. In our case however, we don't actually have access to such systems. For this reason, the remote data exists in an SQLite database like our local one.

\subsection{Subsystem Interface Specifications} \label{subsystem interface}

Specification of the software interfaces between the subsystems,
i.e. specific messages (or function calls) that are exchanged by the subsystems.
These are also often called ``Module Interface Specifications''.
Description of the parameters to be passed into these function calls in order to have a service fulfilled,
including valid and invalid ranges of values.
Each subsystem interface must be presented in a separate subsection.

*Note: The above is a description of what to provide. Need to edit into our own
\subsubsection{View Controllers}

We herewith present the view controllers, and the services they access.
\paragraph{SignUpController Interfaces}

Below are the different models and services used in the SignUpController view.

\begin{figure}[H]
\includegraphics[width=\linewidth]{subsystem-interface-signup-controller.png}
\end{figure}

\clearpage

\paragraph{LoginController Interfaces}

Below are the different models and services used in the LoginController view.

\begin{figure}[H]
\includegraphics[width=\linewidth]{subsystem-interface-login-controller.png}
\end{figure}

\clearpage

\paragraph{AcountDetailsController Interfaces}

Below are the different models and services used in the AccountDetailsController view.

\begin{figure}[H]
\includegraphics[width=\linewidth]{account-details-controller.png}
\end{figure}

\clearpage

\paragraph{AccountListController Interfaces}

Below are the different models and services used in the AccountListController view.

\begin{figure}[H]
\includegraphics[width=\linewidth]{subsystem-interface-AccountListController.jpg}
\end{figure}

\clearpage

\section{Detailed Design} \label{sec:detail}

The myMoney system architecture is designed to be easily modified because of the low coupling between the modules. This was done with interfaces and auto injection of dependencies in classes. Each service package has a Module class designed to bind and provide an implementation to an interface. This way, classes are never instantiated directly into each other, but injected. This design pattern is useful because a change in implementation is as simple as creating a new class and change the module binding. The classes that use it and the tests should in no way be changed. Mocking classes for test purposes is also much easier.

As a side note, we noticed that merge conflicts using git were much less likely to happen because we can each work on different parts of the system without modifying another module.

The tool used for this purpose is Dagger version 2.

\subsection{Class Diagram}

In this section we provide the class diagram of our system, useful for the system developers and testers.This is an in depth look at all of the classes within our system see figure \ref{fig:class-diagram} below If a term is unclear, view section \ref{glossary} for the glossary.

\begin{figure}[H]
\includegraphics[width=\linewidth]{ClassDiagram1.png}
\end{figure}

\begin{figure}[H]
\includegraphics[width=\linewidth]{ClassDiagram2.png}
\end{figure}

\begin{figure}[H]
\includegraphics[width=\linewidth]{ClassDiagram3.png}
\end{figure}

\begin{figure}[H]
\includegraphics[width=\linewidth]{ClassDiagram4.png}
\end{figure}

\begin{figure}[H]
\includegraphics[width=\linewidth]{ClassDiagram5.png}
\caption{Class Diagram}
\label{fig:class-diagram}
\end{figure}

\clearpage

\subsection{Classes}

%Start of com.github.comp354project%
\begin{table}[H]
\centering
\caption{Interface ApplicationComponent}
\label{my-label}
\resizebox{\textwidth}{!}{%
\begin{tabular}{|l|l|l|l|l|}
\hline
\textbf{Class Name}               & \multicolumn{4}{l|}{com.github.comp354project.ApplicationComponent}                                                                                            \\ \hline
\textbf{Type}                     & \multicolumn{4}{l|}{Interface}                                                                                                                                 \\ \hline
\textbf{Inherits}                 & \multicolumn{4}{l|}{N/A}                                                                                                                                       \\ \hline
\textbf{Implements}               & \multicolumn{4}{l|}{N/A}                                                                                                                                       \\ \hline
\textbf{Description}              & \multicolumn{4}{l|}{Class dependencies can be injected into the classes defined in the inject methods. This class is used for the Dagger2 injection framework} \\ \hline
\textbf{Attributes}               & \textbf{Visibility}   & \textbf{Data Type}                                                & \textbf{Name}      & \textbf{Description}                          \\ \hline
None                              &                       &                                                                   &                    &                                               \\ \hline
\multirow{7}{*}{\textbf{Methods}} & \textbf{Visibility}   & \textbf{Name}                                                     & \textbf{Returns}   & \textbf{Description}                          \\ \cline{2-5}
                                  & public                & inject(MyMoneyApplication myMoneyApplication)                     & void               & Injector for MyMoneyApplication class         \\ \cline{2-5}
                                  & public                & inject(LoginController loginController)                           & void               & Injector for the LoginController class        \\ \cline{2-5}
                                  & public                & inject(AccountListController accountListController)               & void               & Injector for the AccountListController class  \\ \cline{2-5}
                                  & public                & inject(SignUpController signUpController)                         & void               & Injector for the SignUpController             \\ \cline{2-5}
                                  & public                & inject(TransactionTableController tableController)                & void               & Injector for the TransactionTableController   \\ \cline{2-5}
                                  & public                & inject(UpdateUserAccountController updateUserAccountController)   & void               & Injector for the UpdateUserAccountController  \\ \hline
\end{tabular}
}
\end{table}

\begin{table}[H]
\centering
\caption{Class BusinessRulesConstants}
\label{my-label}
\resizebox{\textwidth}{!}{%
\begin{tabular}{|l|l|l|l|l|}
\hline
\textbf{Class Name}  & \multicolumn{4}{l|}{com.github.comp354project.BusinessRulesConstants}                               \\ \hline
\textbf{Type}        & \multicolumn{4}{l|}{Class}                                                                          \\ \hline
\textbf{Inherits}    & \multicolumn{4}{l|}{N/A}                                                                            \\ \hline
\textbf{Implements}  & \multicolumn{4}{l|}{N/A}                                                                            \\ \hline
\textbf{Description} & \multicolumn{4}{l|}{Contains business rules configuration for validators}                           \\ \hline
\textbf{Attributes}  & \textbf{Visibility} & \textbf{Data Type} & \textbf{Name}         & \textbf{Description}             \\ \hline
                     & public              & Integer            & USERNAME\_MIN\_LENGTH & The minimum length of a username \\ \hline
                     & public              & Integer            & USERNAME\_MAX\_LENGTH & The maximum length of a username \\ \hline
                     & public              & Integer            & PASSWORD\_MIN\_LENGTH & The minimum length of a password \\ \hline
                     & public              & Integer            & PASSWORD\_MAX\_LENGTH & The maximum length of a password \\ \hline
                     & public              & Integer            & CATEGORY\_MIN\_LENGTH & The minimum length of a category \\ \hline
                     & public              & Integer            & CATEGORY\_MAX\_LENGTH & The maximum length of a category \\ \hline
\textbf{Methods}     & \textbf{Visibility} & \textbf{Name}      & \textbf{Returns}      & \textbf{Description}             \\ \hline
None                 &                     &                    &                       &                                  \\ \hline
\end{tabular}
}
\end{table}

\begin{table}[H]
\centering
\caption{Class Main}
\label{my-label}
\begin{tabular}{|l|l|l|l|l|}
\hline
\textbf{Class Name}  & \multicolumn{4}{l|}{com.github.comp354project.Main}                                                                                              \\ \hline
\textbf{Type}        & \multicolumn{4}{l|}{Class}                                                                                                                       \\ \hline
\textbf{Inherits}    & \multicolumn{4}{l|}{N/A}                                                                                                                         \\ \hline
\textbf{Implements}  & \multicolumn{4}{l|}{N/A}                                                                                                                         \\ \hline
\textbf{Description} & \multicolumn{4}{l|}{Launches the application}                                                                                                    \\ \hline
\textbf{Attributes}  & \textbf{Visibility} & \textbf{Data Type}      & \textbf{Name}    & \textbf{Description}                                                          \\ \hline
None                 &                     &                         &                  &                                                                               \\ \hline
\textbf{Methods}     & \textbf{Visibility} & \textbf{Name}           & \textbf{Returns} & \textbf{Description}                                                          \\ \hline
                     & public              & main(String{[}{]} args) & void             & \begin{tabular}[c]{@{}l@{}}The entry point of \\ the application\end{tabular} \\ \hline
\end{tabular}
\end{table}


\begin{table}[H]
\centering
\caption{Class MyMoneyApplication}
\label{my-label}
\resizebox{\textwidth}{!}{%
\begin{tabular}{|l|l|l|l|l|}
\hline
\textbf{Class Name}  & \multicolumn{4}{l|}{com.github.comp354project.MyMoneyApplication}                                                                                                                                                                         \\ \hline
\textbf{Type}        & \multicolumn{4}{l|}{Class}                                                                                                                                                                                                                \\ \hline
\textbf{Inherits}    & \multicolumn{4}{l|}{Application}                                                                                                                                                                                                          \\ \hline
\textbf{Implements}  & \multicolumn{4}{l|}{N/A}                                                                                                                                                                                                                  \\ \hline
\textbf{Description} & \multicolumn{4}{l|}{Entry point for the GUI of the application}                                                                                                                                                                           \\ \hline
\textbf{Attributes}  & \textbf{Visibility} & \textbf{Data Type}                                            & \textbf{Name}      & \textbf{Description}                                                                                                           \\ \hline
                     & private             & Logger                                                        & logger             & Logs event information                                                                                                         \\ \hline
                     & public              & MyMoneyApplication                                            & application        & The GUI entry point variable                                                                                                   \\ \hline
                     & protected           & SessionManager                                                & sessionManager     & Manages user sessions                                                                                                          \\ \hline
                     & private             & ApplicationComponent                                          & component          & Used to instantiate and inject classes                                                                                         \\ \hline
                     & private             & Stage                                                         & primaryStage       & Used to display the GUI                                                                                                        \\ \hline
\textbf{Methods}     & \textbf{Visibility} & \textbf{Name}                                                 & \textbf{Returns}   & \textbf{Description}                                                                                                           \\ \hline
                     & public              & MyMoneyApplication                                            & MyMoneyApplication & \begin{tabular}[c]{@{}l@{}}Constructs the class. \\ Initializes an ApplicationComponent\\ for depedency injection\end{tabular} \\ \hline
                     & public              & getScene()                                                    & Scene              & Returns the current scene                                                                                                      \\ \hline
                     & public              & start(Stage primaryStage)                                     & void               & \begin{tabular}[c]{@{}l@{}}Displays the first GUI\\ when the application\\ launches\end{tabular}                               \\ \hline
                     & private             & updateStage(String fxml, String title, int width, int height) & T                  & Updates the current view                                                                                                       \\ \hline
                     & private             & setStageTitle(String title)                                   & void               & Sets the view's title                                                                                                          \\ \hline
                     & public              & displayLogin()                                                & void               & Displays the login view                                                                                                        \\ \hline
                     & public              & displaySignUp()                                               & void               & Displays the sign up view                                                                                                      \\ \hline
                     & public              & displayAccounts()                                             & void               & Displays the user accounts view                                                                                                \\ \hline
                     & public              & displayUpdateUser()                                           & void               & Displays the update user view                                                                                                  \\ \hline
                     & public              & displayAccountDetails(Account account)                        & void               & Displays the account details view                                                                                              \\ \hline
                     & public              & displayAllAccountDetails(List accounts)                       & void               & Displays all accounts details view                                                                                             \\ \hline
\end{tabular}%
}
\end{table}
%End of com.github.comp354project%

%Start of com.github.comp354project.service.account

\begin{table}[H]
\centering
\caption{Class Account}
\label{my-label}
\resizebox{\textwidth}{!}{%
\begin{tabular}{|l|l|l|l|l|}
\hline
\textbf{Class Name}                  & \multicolumn{4}{l|}{com.github.comp354project.service.account.Account}                                                                             \\ \hline
\textbf{Type}                        & \multicolumn{4}{l|}{Class}                                                                                                                \\ \hline
\textbf{Inherits}                    & \multicolumn{4}{l|}{N/A}                                                                                                                   \\ \hline
\textbf{Implements}                  & \multicolumn{4}{l|}{N/A}                                                                                                                   \\ \hline
\textbf{Description}                 & \multicolumn{4}{l|}{Used to hold the account information of the user}                                                                      \\ \hline
\multirow{6}{*}{\textbf{Attributes}} & \textbf{Visibility} & \textbf{Data Type}                                & \textbf{Name}    & \textbf{Description}                          \\ \cline{2-5}
                                     & private             & Integer                                           & ID               & bank account identification number            \\ \cline{2-5}
                                     & private             & String                                            & type             & type of bank account (chequing, savings, ect) \\ \cline{2-5}
                                     & private             & Double                                            & balance          & Amount inside the account                     \\ \cline{2-5}
                                     & private             & User                                              & user             & name of the user                              \\ \cline{2-5}
                                     & private             & ForeignCollection\textless Transaction\textgreater & transactions     & transaction object                            \\ \hline
\multirow{2}{*}{\textbf{Methods}}    & \textbf{Visibility} & \textbf{Name}                                     & \textbf{Returns} & \textbf{Description}                          \\ \cline{2-5}
                                     & none                & none                                              & none             & none                                          \\ \hline
\end{tabular}%
}
\end{table}


\begin{table}[H]
\centering
\caption{Class AccountService}
\label{my-label}
\resizebox{\textwidth}{!}{%
\begin{tabular}{|l|l|l|l|l|}
\hline
\textbf{Class Name}                  & \multicolumn{4}{l|}{com.github.comp354project.service.account.AccountService}                                                                                                                                   \\ \hline
\textbf{Type}                        & \multicolumn{4}{l|}{Class}                                                                                                                                                                                     \\ \hline
\textbf{Inherits}                    & \multicolumn{4}{l|}{N/A}                                                                                                                                                                                        \\ \hline
\textbf{Implements}                  & \multicolumn{4}{l|}{IAccountService}                                                                                                                                                                            \\ \hline
\textbf{Description}                 & \multicolumn{4}{l|}{Class used to request information from the bank database in order to add or delete an account to myMoney application}                                                                       \\ \hline
\multirow{5}{*}{\textbf{Attributes}} & \textbf{Visibility} & \textbf{Data Type}                          & \textbf{Name}        & \textbf{Description}                                                                                                 \\ \cline{2-5}
                                     & private             & Logger                                      & logger               & logger object attribute used to keep track of errors                                                                 \\ \cline{2-5}
                                     & private             & Dao\textless Transaction,Integer\textgreater & transactionDao       & Dao object used to query the database                                                                                \\ \cline{2-5}
                                     & private             & Dao\textless User,Integer\textgreater        & userDao              & Dao object used for quering the database                                                                             \\ \cline{2-5}
                                     & private             & IRemoteAccountService                       & remoteAccountService & attribute used to access database                                                                                    \\ \hline
\multirow{5}{*}{\textbf{Methods}}    & \textbf{Visibility} & \textbf{Name}                               & \textbf{Returns}     & \textbf{Description}                                                                                                 \\ \cline{2-5}
                                     & public              & addAccount                                  & Account              & method to request bank information from the database                                                                 \\ \cline{2-5}
                                     & public              & deleteAccount                               & void                 & method to delete a particular account from myMoney application                                                       \\ \cline{2-5}
                                     & public              & transform                                   & Account              & method to create the appropriate banking info to display for the myMoney app based on the retrieved banking info     \\ \cline{2-5}
                                     & public              & Transaction                                 & transform            & method to create the appropriate transaction info to display for the myMoney app based on the retrieved banking info \\ \hline
\end{tabular}%
}
\end{table}



\begin{table}[H]
\centering
\caption{Class AccountServiceModule}
\label{my-label}
\resizebox{\textwidth}{!}{%
\begin{tabular}{|l|l|l|l|l|}
\hline
\textbf{Class Name}                  & \multicolumn{4}{l|}{com.github.comp354project.service.account.AccountServiceModule}                     \\ \hline
\textbf{Type}                        & \multicolumn{4}{l|}{Class}                                                                                \\ \hline
\textbf{Inherits}                    & \multicolumn{4}{l|}{N/A}                                                                                \\ \hline
\textbf{Implements}                  & \multicolumn{4}{l|}{N/A}                                                                                \\ \hline
\textbf{Description}                 & \multicolumn{4}{l|}{used to return need objects for account and transaction needs}                      \\ \hline
\multirow{2}{*}{\textbf{Attributes}} & \textbf{Visibility} & \textbf{Data Type}        & \textbf{Name}      & \textbf{Description}             \\ \cline{2-5}
                                     & None                & none                      & none               & none                             \\ \hline
\multirow{3}{*}{\textbf{Methods}}    & \textbf{Visibility} & \textbf{Name}             & \textbf{Returns}   & \textbf{Description}             \\ \cline{2-5}
                                     & public              & provideTransactionService & transactionService & return transactionService Object \\ \cline{2-5}
                                     & public              & provideAccountService     & accountService     & returns accountService Object    \\ \hline
\end{tabular}%
}
\end{table}


\begin{table}[H]
\centering
\caption{Interface IAccountService}
\label{my-label}
\resizebox{\textwidth}{!}{%
\begin{tabular}{|l|l|l|l|l|}
\hline
\textbf{Class Name}               & \multicolumn{4}{l|}{com.github.comp354project.service.account.IAccountService}     \\ \hline
\textbf{Type}                     & \multicolumn{4}{l|}{Interface}                                                     \\ \hline
\textbf{Inherits}                 & \multicolumn{4}{l|}{N/A}                                                           \\ \hline
\textbf{Implements}               & \multicolumn{4}{l|}{N/A}                                                           \\ \hline
\textbf{Description}              & \multicolumn{4}{l|}{interface class for adding and deleting an account}            \\ \hline
\textbf{Attributes}               & \textbf{Visibility} & \textbf{Data Type} & \textbf{Name}    & \textbf{Description} \\ \hline
None                              & None                & None               & none             & none                 \\ \hline
\multirow{3}{*}{\textbf{Methods}} & \textbf{Visibility} & \textbf{Name}      & \textbf{Returns} & \textbf{Description} \\ \cline{2-5}
                                  & N/A                 & addAccount         & N/A              & none                 \\ \cline{2-5}
                                  & N/A                 & deleteAccount      & N/A              & none                 \\ \hline
\end{tabular}%
}
\end{table}


\begin{table}[H]
\centering
\caption{Interface ITransactionService}
\label{my-label}
\resizebox{\textwidth}{!}{%
\begin{tabular}{|l|l|l|l|l|}
\hline
\textbf{Class Name}               & \multicolumn{4}{l|}{com.github.comp354project.service.account.ITransactionService}        \\ \hline
\textbf{Type}                     & \multicolumn{4}{l|}{Interface}                                                            \\ \hline
\textbf{Inherits}                 & \multicolumn{4}{l|}{N/A}                                                                  \\ \hline
\textbf{Implements}               & \multicolumn{4}{l|}{N/A}                                                                  \\ \hline
\textbf{Description}              & \multicolumn{4}{l|}{interface class to updating transactions based on categories}         \\ \hline
\textbf{Attributes}               & \textbf{Visibility} & \textbf{Data Type}        & \textbf{Name}    & \textbf{Description} \\ \hline
None                              & None                & None                      & None             & None                 \\ \hline
\multirow{2}{*}{\textbf{Methods}} & \textbf{Visibility} & \textbf{Name}             & \textbf{Returns} & \textbf{Description} \\ \cline{2-5}
                                  & N/A                 & updateTransactionCategory & Transaction      & N/A                  \\ \hline
\end{tabular}%
}
\end{table}


\begin{table}[H]
\centering
\caption{Class Transaction}
\label{my-label}
\resizebox{\textwidth}{!}{%
\begin{tabular}{|l|l|l|l|l|}
\hline
\textbf{Class Name}                  & \multicolumn{4}{l|}{com.github.comp354project.service.account.Transaction}                       \\ \hline
\textbf{Type}                        & \multicolumn{4}{l|}{Class}                                                                         \\ \hline
\textbf{Inherits}                    & \multicolumn{4}{l|}{N/A}                                                                         \\ \hline
\textbf{Implements}                  & \multicolumn{4}{l|}{N/A}                                                                         \\ \hline
\textbf{Description}                 & \multicolumn{4}{l|}{Class used to contain the attributes needed to hold a transaction's details} \\ \hline
\multirow{8}{*}{\textbf{Attributes}} & \textbf{Visibility}  & \textbf{Data Type}  & \textbf{Name}     & \textbf{Description}            \\ \cline{2-5}
                                     & private              & Integer             & date              & date of a transaction           \\ \cline{2-5}
                                     & private              & Double              & amount            & dollar amount of a transaction  \\ \cline{2-5}
                                     & private              & String              & type              & the type of a transaction       \\ \cline{2-5}
                                     & private              & String              & category          & the category of a transaction   \\ \cline{2-5}
                                     & private              & Integer             & sourceID          & ID number                       \\ \cline{2-5}
                                     & private              & Integer             & destinationID     & ID number                       \\ \cline{2-5}
                                     & private              & Account             & account           & name of the account             \\ \hline
\multirow{2}{*}{\textbf{Methods}}    & \textbf{Visibility}  & \textbf{Name}       & \textbf{Returns}  & \textbf{Description}            \\ \cline{2-5}
                                     & None                 & None                & None              & None                            \\ \hline
\end{tabular}%
}
\end{table}


\begin{table}[H]
\centering
\caption{Class TransactionService}
\label{my-label}
\resizebox{\textwidth}{!}{%
\begin{tabular}{|l|l|l|l|l|}
\hline
\textbf{Class Name}                  & \multicolumn{4}{l|}{com.github.comp354project.service.account.TransactionService}                                                              \\ \hline
\textbf{Type}                        & \multicolumn{4}{l|}{Class}                                                                                                                       \\ \hline
\textbf{Inherits}                    & \multicolumn{4}{l|}{N/A}                                                                                                                       \\ \hline
\textbf{Implements}                  & \multicolumn{4}{l|}{ITransactionService}                                                                                                       \\ \hline
\textbf{Description}                 & \multicolumn{4}{l|}{class used to help with transaction changes}                                                                               \\ \hline
\multirow{4}{*}{\textbf{Attributes}} & \textbf{Visibility} & \textbf{Data Type}                          & \textbf{Name}     & \textbf{Description}                                   \\ \cline{2-5}
                                     & private             & Logger                                      & logger            & object used to interact with TransactionService class  \\ \cline{2-5}
                                     & private             & Dao\textless Transaction,Integer\textgreater & transactionDao    & object used to perform methods related to transactions \\ \cline{2-5}
                                     & private             & ICategoryNameValidator                      & categoryValidator & object used to validate if a category is correct       \\ \hline
\multirow{3}{*}{\textbf{Methods}}    & \textbf{Visibility} & \textbf{Name}                               & \textbf{Returns}  & \textbf{Description}                                   \\ \cline{2-5}
                                     & public              & TransactionService                          & N/A               & constructor                                            \\ \cline{2-5}
                                     & public              & updateTransactionCategory                   & Transaction       & used to update a specific transation                   \\ \hline
\end{tabular}%
}
\end{table}

%End of com.github.comp354project.service.account

% Start of com.github.comp354project.service.dao

\begin{table}[H]
\centering
\caption{Class DaoModule}
\label{my-label}
\resizebox{\textwidth}{!}{%
\begin{tabular}{|l|l|l|l|l|}
\hline
\textbf{Class Name}  & \multicolumn{4}{l|}{com.github.comp354project.service.dao.DaoModule}                                                                                                                                                                 \\ \hline
\textbf{Type}        & \multicolumn{4}{l|}{Class}                                                                                                                                                                                                           \\ \hline
\textbf{Inherits}    & \multicolumn{4}{l|}{N/A}                                                                                                                                                                                                             \\ \hline
\textbf{Implements}  & \multicolumn{4}{l|}{N/A}                                                                                                                                                                                                             \\ \hline
\textbf{Description} & \multicolumn{4}{l|}{DAO module to bind interfaces to their interfaces and provide them to the classes that require them}                                                                                                             \\ \hline
\textbf{Attributes}  & \textbf{Visibility} & \textbf{Data Type}                                              & \textbf{Name}                                  & \textbf{Description}                                                                        \\ \hline
                     & private             & Logger                                                          & logger                                         & Logs event information                                                                      \\ \hline
\textbf{Methods}     & \textbf{Visibility} & \textbf{Name}                                                   & \textbf{Returns}                               & \textbf{Description}                                                                        \\ \hline
                     & public              & provideRemoteAccountDao(IConnectionProvider connectionProvider) & Dao\textless RemoteAccount, Integer\textgreater & \begin{tabular}[c]{@{}l@{}}Returns the implementation \\ of a RemoteAccountDao\end{tabular} \\ \hline
                     & public              & provideAccountDao(IConnectionProvider connectionProvider)       & Dao\textless Account, Integer\textgreater       & \begin{tabular}[c]{@{}l@{}}Returns the implementation\\ of an AccountDao\end{tabular}       \\ \hline
                     & public              & provideTransactionDao(IConnectionProvider connectionProvider)   & Dao\textless Transaction, Integer\textgreater   & \begin{tabular}[c]{@{}l@{}}Returns the implementation\\ of a TransactionDao\end{tabular}    \\ \hline
                     & public              & provideUserDao(IConnectionProvider connectionProvider)          & Dao\textless User, Integer\textgreater          & \begin{tabular}[c]{@{}l@{}}Returns the implementation\\ of a UserDao\end{tabular}           \\ \hline
\end{tabular}%
}
\end{table}

%end of com.github.comp354project.service.dao

% Start of com.github.comp354project.service.sqlite

\begin{table}[H]
\centering
\caption{Class ConnectionModule}
\label{my-label}
\resizebox{\textwidth}{!}{%
\begin{tabular}{|l|l|l|l|l|}
\hline
\textbf{Class Name}  & \multicolumn{4}{l|}{com.github.comp354project.service.sqlite.ConnectionModule}                                                                                                                      \\ \hline
\textbf{Type}        & \multicolumn{4}{l|}{Class}                                                                                                                                                                          \\ \hline
\textbf{Inherits}    & \multicolumn{4}{l|}{N/A}                                                                                                                                                                            \\ \hline
\textbf{Implements}  & \multicolumn{4}{l|}{N/A}                                                                                                                                                                            \\ \hline
\textbf{Description} & \multicolumn{4}{l|}{Module that creates a connection to the database}                                                                                                                               \\ \hline
\textbf{Attributes}  & \textbf{Visibility} & \textbf{Data Type}                                       & \textbf{Name}       & \textbf{Description}                                                                         \\ \hline
None                 &                     &                                                          &                     &                                                                                              \\ \hline
\textbf{Methods}     & \textbf{Visibility} & \textbf{Name}                                            & \textbf{Returns}    & \textbf{Description}                                                                         \\ \hline
                     & protected           & provideConnection(ConnectionProvider connectionProvider) & IConnectionProvider & \begin{tabular}[c]{@{}l@{}}Returns the implementation\\ of a ConnectionProvider\end{tabular} \\ \hline
\end{tabular}%
}
\end{table}

\begin{table}[H]
\centering
\caption{Class ConnectionProvider}
\label{my-label}
\resizebox{\textwidth}{!}{%
\begin{tabular}{|l|l|l|l|l|}
\hline
\textbf{Class Name}  & \multicolumn{4}{l|}{com.github.comp354project.service.sqlite.ConnectionProvider}                                                                    \\ \hline
\textbf{Type}        & \multicolumn{4}{l|}{Class}                                                                                                                          \\ \hline
\textbf{Inherits}    & \multicolumn{4}{l|}{N/A}                                                                                                                            \\ \hline
\textbf{Implements}  & \multicolumn{4}{l|}{IConnectionProvider}                                                                                                            \\ \hline
\textbf{Description} & \multicolumn{4}{l|}{Instatiates a connection to an SQLite database}                                                                                 \\ \hline
\textbf{Attributes}  & \textbf{Visibility} & \textbf{Data Type}    & \textbf{Name}        & \textbf{Description}                                                           \\ \hline
                     & private             & Logger                & logger               & Logs events                                                                    \\ \hline
\textbf{Methods}     & \textbf{Visibility} & \textbf{Name}         & \textbf{Returns}     & \textbf{Description}                                                           \\ \hline
                     & public              & ConnectionProvider()  & ConnectionProvider   & Constructs the class                                                           \\ \hline
                     & public              & getConnectionSource() & JdbcConnectionSource & \begin{tabular}[c]{@{}l@{}}Returns a database connection\\ source\end{tabular} \\ \hline
\end{tabular}%
}
\end{table}

\begin{table}[H]
\centering
\caption{Interface IConnectionProvider}
\label{my-label}
\documentclass[12pt]{article}

\usepackage{color}
\usepackage{nth}
\usepackage{enumitem}
\usepackage{booktabs}
\usepackage{tabularx}
\usepackage{hyperref}
\usepackage[pdftex]{graphicx}
\pagestyle{empty}
\setcounter{secnumdepth}{2}
\usepackage{float}
\usepackage{multirow}

\pagestyle{empty}
\setcounter{secnumdepth}{2}

\topmargin=0cm
\oddsidemargin=0cm
\textheight=22.0cm
\textwidth=16cm
\parindent=0cm
\parskip=0.15cm
\topskip=0truecm
\raggedbottom
\abovedisplayskip=3mm
\belowdisplayskip=3mm
\abovedisplayshortskip=0mm
\belowdisplayshortskip=2mm
\normalbaselineskip=12pt
\normalbaselines

\begin{document}

\vspace*{0.5in}
\centerline{\bf\Large COMP 354}
\centerline{\bf\Large Design Document for myMoney}

\vspace*{0.5in}
\centerline{\bf\Large Team PA-PK}

\vspace*{0.5in}
\centerline{\today}

\vspace*{1.5in}
\begin{table}[htbp]
\caption{Team}
\begin{center}
\begin{tabular}{|r | c|}
\hline
Name & ID Number \\
\hline\hline
Anne-Laure Ehresmann & 27858906 \\
\hline
Marc-Antoine Dube & 40029307 \\
\hline
Kadeem Caines & 26343600 \\
\hline
Abdel Rahman Jawhar & 27192142 \\
\hline
Keith Dion & 40036340 \\
\hline
Hrachya Hakobyan & 40041555 \\
\hline
Andrew-Smith & 40034936 \\
\hline
Dongyu Chen & 27241909 \\
\hline
Yauheni Karaniuk & 40005680 \\
\hline
Renny Xu & 40005262\\
\hline
Wei Wang & 40041116 \\
\hline
\end{tabular}
\end{center}
\end{table}

\clearpage

\tableofcontents
\listoffigures
\listoftables

\clearpage


\section{Introduction and Purpose}

The goal of this document is to define the design for the desktop application myMoney. The majority of the design decisions have been taken     with the Requirements document in mind, one may thus want to look at this document first to have a clear picture of the problem in mind as well as the requirements demanded for the solution. This document presents an implementation of a possible solution to answer this problem. Its design is is outlined through an Architectural Design (AD), a Detailed design (DD) and Dynamic Design Scenarios (DDS) for the application. The AD focuses on high-level project decomposition, the DD describes the overarching system design (which includes the UML design, divided into multiple subsections), and the DDS displays how the subsystems interact with one another in order to produce system-level services. This document may thus be used to plan, coordinate, and guide the development of the software, estimate and allocate necessary resources for proper execution, and  then actually implement the software for the system. It seeks, above all, to serve as a precise and stable reference throughout development.

\section{Scope}

This document contains everything to do with the development decisions and design of the system, all of which are derived from the requirements, which are not described in this document. Also not included in here is any testing of the system, which verifies that the requirements are met. It is merely a blueprint for a system that should, in theory, successfully pass any tests that would be done in correspondence with the requirements.

\section{Architectural Design} \label{sec:arch}

The myMoney application uses the Model-View-Controller (MVC) architectural pattern at its core.

The \textit{view} is implemented through a JAVAFX front-end interface, which consists mostly of hard-coded tables structures or menus, which are then populated with table entries by the model, as requested by the view controllers. The view is interactive, and reports any events triggered by the user to the view controllers, which then pass the requests to the model controllers (services), which handle the modification of the model. Once updated, the model then passes its changes back to the view controllers, to update the information displayed to the user. A more in-depth view of the interaction between the user and the view can be seen in the dynamic models, in section \ref{Dynamic Models}.

The \textit{controllers} are the most complex part of the system. They are organised in a layered structure of services, which each handle a different sector of the application (session management, account services, user services, etc). All services perform their own validation, whether it is for implementing the business rules, or simply ensuring expected application behaviour (no null objects, caught exceptions...). Once this validation has been verified, the controllers passes the actually requested change to the model. Examples of such services, their intercommunication, and their validation, is explained more in-depth in section \ref{subsystem interface}.

The \textit{model}, which includes the back-end connection to the database, is composed of data access objects (DAOs), which are used to apply edits to the database, after the upper layers of the controllers have verified the validity of the calls. Our system actually employs two databases; The first, a local database holding user info, bank account info, and transaction info, and the second, a "remote" database (also local, but acts as if it were remote) used to simulate the bank servers. The only time the second database is actually accessed is when the user first adds a bank account. In this case, our view receives his input, passes it to the controls, which then makes requests for information to the "remote" database through the use of the services handling remote communication. These services receive, from the DAOs, a serialisation of the "remote" account, which it then translates into usable data for the local database. Once this has been successfully executed, the model triggers a view update, wherein the user can see his newly requested additions. All other events triggered by the user have no need of the remote database, and simply employ a series of communications between the controllers and the services handling local database. See section \ref{subsystem interface} for more details on these services and the databases.

\subsection{Architectural Diagram}

\begin{figure}[H]
\includegraphics[width=\linewidth]{package-diagram.png}
\end{figure}

We herewith present the architectural diagram of the design presented above, and provide an explanation of each package (but not for the subsystems within each package). A succint and precise description of each subsystem is available in the next section.

Notice first that MVC's style forces a clear separation of concerns, and thus emphasises a great amount of intercommunication between each section. To embrace this, using interfaces for each component presented above will clearly separate the implementation from the structure of the module, which eases parallel addition, modification, or testing of the system.

The com.github.comp354project.service package is the main subsystem of our project. As mentioned above, it is organised in a layered manner, wherein each layer handles its own services, and use the services of the layer below it within worrying about that layer's implementation. Data validation and processing is offered by each service: The account service, for example, validate calls to add or delete bank accounts, edit a transaction's category, or query for specific accounts. It does this by querying the database using an account DAO, and ensuring data integrity and validity (with regards to the business rules). It does not, however, worry about user authorisation, and simply assumes the layer above it (The user service) will have handled it. The com.github.comp354project.viewController calls services within this package to update the view and the model.

The com.github.comp354project.viewController package contains a number of controllers for each different view. They are the ones handling the requests from the users, which they pass to the services in the package described above. They are also the ones who pass any errors from the services to the views, mostly to be used for testing and alerting the user of any problems that might have occured.

The com.github.comp354project.service package.account.remote package is a subsystem to our services which mocks an API call to remote servers of banks or credit card companies. In our case however, we don't actually have access to such systems. For this reason, the remote data exists in an SQLite database like our local one.

\subsection{Subsystem Interface Specifications} \label{subsystem interface}

Specification of the software interfaces between the subsystems,
i.e. specific messages (or function calls) that are exchanged by the subsystems.
These are also often called ``Module Interface Specifications''.
Description of the parameters to be passed into these function calls in order to have a service fulfilled,
including valid and invalid ranges of values.
Each subsystem interface must be presented in a separate subsection.

*Note: The above is a description of what to provide. Need to edit into our own
\subsubsection{View Controllers}

We herewith present the view controllers, and the services they access.
\paragraph{SignUpController Interfaces}

Below are the different models and services used in the SignUpController view.

\begin{figure}[H]
\includegraphics[width=\linewidth]{subsystem-interface-signup-controller.png}
\end{figure}

\clearpage

\paragraph{LoginController Interfaces}

Below are the different models and services used in the LoginController view.

\begin{figure}[H]
\includegraphics[width=\linewidth]{subsystem-interface-login-controller.png}
\end{figure}

\clearpage

\paragraph{AcountDetailsController Interfaces}

Below are the different models and services used in the AccountDetailsController view.

\begin{figure}[H]
\includegraphics[width=\linewidth]{account-details-controller.png}
\end{figure}

\clearpage

\paragraph{AccountListController Interfaces}

Below are the different models and services used in the AccountListController view.

\begin{figure}[H]
\includegraphics[width=\linewidth]{subsystem-interface-AccountListController.png}
\end{figure}

\clearpage

\section{Detailed Design} \label{sec:detail}

The myMoney system architecture is designed to be easily modified because of the low coupling between the modules. This was done with interfaces and auto injection of dependencies in classes. Each service package has a Module class designed to bind and provide an implementation to an interface. This way, classes are never instantiated directly into each other, but injected. This design pattern is useful because a change in implementation is as simple as creating a new class and change the module binding. The classes that use it and the tests should in no way be changed. Mocking classes for test purposes is also much easier.

As a side note, we noticed that merge conflicts using git were much less likely to happen because we can each work on different parts of the system without modifying another module.

The tool used for this purpose is Dagger version 2.

\subsection{Class Diagram}

In this section we provide the class diagram of our system, useful for the system developers and testers.This is an in depth look at all of the classes within our system see figure \ref{fig:class-diagram} below If a term is unclear, view section \ref{glossary} for the glossary.

\begin{figure}[H]
\includegraphics[width=\linewidth]{ClassDiagram1.png}
\end{figure}

\begin{figure}[H]
\includegraphics[width=\linewidth]{ClassDiagram2.png}
\end{figure}

\begin{figure}[H]
\includegraphics[width=\linewidth]{ClassDiagram3.png}
\end{figure}

\begin{figure}[H]
\includegraphics[width=\linewidth]{ClassDiagram4.png}
\end{figure}

\begin{figure}[H]
\includegraphics[width=\linewidth]{ClassDiagram5.png}
\caption{Class Diagram}
\label{fig:class-diagram}
\end{figure}

\clearpage

\subsection{Classes}

%Start of com.github.comp354project%
\begin{table}[H]
\centering
\caption{Interface ApplicationComponent}
\label{my-label}
\resizebox{\textwidth}{!}{%
\begin{tabular}{|l|l|l|l|l|}
\hline
\textbf{Class Name}               & \multicolumn{4}{l|}{com.github.comp354project.ApplicationComponent}                                                                                            \\ \hline
\textbf{Type}                     & \multicolumn{4}{l|}{Interface}                                                                                                                                 \\ \hline
\textbf{Inherits}                 & \multicolumn{4}{l|}{N/A}                                                                                                                                       \\ \hline
\textbf{Implements}               & \multicolumn{4}{l|}{N/A}                                                                                                                                       \\ \hline
\textbf{Description}              & \multicolumn{4}{l|}{Class dependencies can be injected into the classes defined in the inject methods. This class is used for the Dagger2 injection framework} \\ \hline
\textbf{Attributes}               & \textbf{Visibility}   & \textbf{Data Type}                                                & \textbf{Name}      & \textbf{Description}                          \\ \hline
None                              &                       &                                                                   &                    &                                               \\ \hline
\multirow{7}{*}{\textbf{Methods}} & \textbf{Visibility}   & \textbf{Name}                                                     & \textbf{Returns}   & \textbf{Description}                          \\ \cline{2-5}
                                  & public                & inject(MyMoneyApplication myMoneyApplication)                     & void               & Injector for MyMoneyApplication class         \\ \cline{2-5}
                                  & public                & inject(LoginController loginController)                           & void               & Injector for the LoginController class        \\ \cline{2-5}
                                  & public                & inject(AccountListController accountListController)               & void               & Injector for the AccountListController class  \\ \cline{2-5}
                                  & public                & inject(SignUpController signUpController)                         & void               & Injector for the SignUpController             \\ \cline{2-5}
                                  & public                & inject(TransactionTableController tableController)                & void               & Injector for the TransactionTableController   \\ \cline{2-5}
                                  & public                & inject(UpdateUserAccountController updateUserAccountController)   & void               & Injector for the UpdateUserAccountController  \\ \hline
\end{tabular}
}
\end{table}

\begin{table}[H]
\centering
\caption{Class BusinessRulesConstants}
\label{my-label}
\resizebox{\textwidth}{!}{%
\begin{tabular}{|l|l|l|l|l|}
\hline
\textbf{Class Name}  & \multicolumn{4}{l|}{com.github.comp354project.BusinessRulesConstants}                               \\ \hline
\textbf{Type}        & \multicolumn{4}{l|}{Class}                                                                          \\ \hline
\textbf{Inherits}    & \multicolumn{4}{l|}{N/A}                                                                            \\ \hline
\textbf{Implements}  & \multicolumn{4}{l|}{N/A}                                                                            \\ \hline
\textbf{Description} & \multicolumn{4}{l|}{Contains business rules configuration for validators}                           \\ \hline
\textbf{Attributes}  & \textbf{Visibility} & \textbf{Data Type} & \textbf{Name}         & \textbf{Description}             \\ \hline
                     & public              & Integer            & USERNAME\_MIN\_LENGTH & The minimum length of a username \\ \hline
                     & public              & Integer            & USERNAME\_MAX\_LENGTH & The maximum length of a username \\ \hline
                     & public              & Integer            & PASSWORD\_MIN\_LENGTH & The minimum length of a password \\ \hline
                     & public              & Integer            & PASSWORD\_MAX\_LENGTH & The maximum length of a password \\ \hline
                     & public              & Integer            & CATEGORY\_MIN\_LENGTH & The minimum length of a category \\ \hline
                     & public              & Integer            & CATEGORY\_MAX\_LENGTH & The maximum length of a category \\ \hline
\textbf{Methods}     & \textbf{Visibility} & \textbf{Name}      & \textbf{Returns}      & \textbf{Description}             \\ \hline
None                 &                     &                    &                       &                                  \\ \hline
\end{tabular}
}
\end{table}

\begin{table}[H]
\centering
\caption{Class Main}
\label{my-label}
\begin{tabular}{|l|l|l|l|l|}
\hline
\textbf{Class Name}  & \multicolumn{4}{l|}{com.github.comp354project.Main}                                                                                              \\ \hline
\textbf{Type}        & \multicolumn{4}{l|}{Class}                                                                                                                       \\ \hline
\textbf{Inherits}    & \multicolumn{4}{l|}{N/A}                                                                                                                         \\ \hline
\textbf{Implements}  & \multicolumn{4}{l|}{N/A}                                                                                                                         \\ \hline
\textbf{Description} & \multicolumn{4}{l|}{Launches the application}                                                                                                    \\ \hline
\textbf{Attributes}  & \textbf{Visibility} & \textbf{Data Type}      & \textbf{Name}    & \textbf{Description}                                                          \\ \hline
None                 &                     &                         &                  &                                                                               \\ \hline
\textbf{Methods}     & \textbf{Visibility} & \textbf{Name}           & \textbf{Returns} & \textbf{Description}                                                          \\ \hline
                     & public              & main(String{[}{]} args) & void             & \begin{tabular}[c]{@{}l@{}}The entry point of \\ the application\end{tabular} \\ \hline
\end{tabular}
\end{table}


\begin{table}[H]
\centering
\caption{Class MyMoneyApplication}
\label{my-label}
\resizebox{\textwidth}{!}{%
\begin{tabular}{|l|l|l|l|l|}
\hline
\textbf{Class Name}  & \multicolumn{4}{l|}{com.github.comp354project.MyMoneyApplication}                                                                                                                                                                         \\ \hline
\textbf{Type}        & \multicolumn{4}{l|}{Class}                                                                                                                                                                                                                \\ \hline
\textbf{Inherits}    & \multicolumn{4}{l|}{Application}                                                                                                                                                                                                          \\ \hline
\textbf{Implements}  & \multicolumn{4}{l|}{N/A}                                                                                                                                                                                                                  \\ \hline
\textbf{Description} & \multicolumn{4}{l|}{Entry point for the GUI of the application}                                                                                                                                                                           \\ \hline
\textbf{Attributes}  & \textbf{Visibility} & \textbf{Data Type}                                            & \textbf{Name}      & \textbf{Description}                                                                                                           \\ \hline
                     & private             & Logger                                                        & logger             & Logs event information                                                                                                         \\ \hline
                     & public              & MyMoneyApplication                                            & application        & The GUI entry point variable                                                                                                   \\ \hline
                     & protected           & SessionManager                                                & sessionManager     & Manages user sessions                                                                                                          \\ \hline
                     & private             & ApplicationComponent                                          & component          & Used to instantiate and inject classes                                                                                         \\ \hline
                     & private             & Stage                                                         & primaryStage       & Used to display the GUI                                                                                                        \\ \hline
\textbf{Methods}     & \textbf{Visibility} & \textbf{Name}                                                 & \textbf{Returns}   & \textbf{Description}                                                                                                           \\ \hline
                     & public              & MyMoneyApplication                                            & MyMoneyApplication & \begin{tabular}[c]{@{}l@{}}Constructs the class. \\ Initializes an ApplicationComponent\\ for depedency injection\end{tabular} \\ \hline
                     & public              & getScene()                                                    & Scene              & Returns the current scene                                                                                                      \\ \hline
                     & public              & start(Stage primaryStage)                                     & void               & \begin{tabular}[c]{@{}l@{}}Displays the first GUI\\ when the application\\ launches\end{tabular}                               \\ \hline
                     & private             & updateStage(String fxml, String title, int width, int height) & T                  & Updates the current view                                                                                                       \\ \hline
                     & private             & setStageTitle(String title)                                   & void               & Sets the view's title                                                                                                          \\ \hline
                     & public              & displayLogin()                                                & void               & Displays the login view                                                                                                        \\ \hline
                     & public              & displaySignUp()                                               & void               & Displays the sign up view                                                                                                      \\ \hline
                     & public              & displayAccounts()                                             & void               & Displays the user accounts view                                                                                                \\ \hline
                     & public              & displayUpdateUser()                                           & void               & Displays the update user view                                                                                                  \\ \hline
                     & public              & displayAccountDetails(Account account)                        & void               & Displays the account details view                                                                                              \\ \hline
                     & public              & displayAllAccountDetails(List accounts)                       & void               & Displays all accounts details view                                                                                             \\ \hline
\end{tabular}%
}
\end{table}
%End of com.github.comp354project%

%Start of com.github.comp354project.service.account

\begin{table}[H]
\centering
\caption{Class Account}
\label{my-label}
\resizebox{\textwidth}{!}{%
\begin{tabular}{|l|l|l|l|l|}
\hline
\textbf{Class Name}                  & \multicolumn{4}{l|}{com.github.comp354project.service.account.Account}                                                                             \\ \hline
\textbf{Type}                        & \multicolumn{4}{l|}{Class}                                                                                                                \\ \hline
\textbf{Inherits}                    & \multicolumn{4}{l|}{N/A}                                                                                                                   \\ \hline
\textbf{Implements}                  & \multicolumn{4}{l|}{N/A}                                                                                                                   \\ \hline
\textbf{Description}                 & \multicolumn{4}{l|}{Used to hold the account information of the user}                                                                      \\ \hline
\multirow{6}{*}{\textbf{Attributes}} & \textbf{Visibility} & \textbf{Data Type}                                & \textbf{Name}    & \textbf{Description}                          \\ \cline{2-5}
                                     & private             & Integer                                           & ID               & bank account identification number            \\ \cline{2-5}
                                     & private             & String                                            & type             & type of bank account (chequing, savings, ect) \\ \cline{2-5}
                                     & private             & Double                                            & balance          & Amount inside the account                     \\ \cline{2-5}
                                     & private             & User                                              & user             & name of the user                              \\ \cline{2-5}
                                     & private             & ForeignCollection\textless Transaction\textgreater & transactions     & transaction object                            \\ \hline
\multirow{2}{*}{\textbf{Methods}}    & \textbf{Visibility} & \textbf{Name}                                     & \textbf{Returns} & \textbf{Description}                          \\ \cline{2-5}
                                     & none                & none                                              & none             & none                                          \\ \hline
\end{tabular}%
}
\end{table}


\begin{table}[H]
\centering
\caption{Class AccountService}
\label{my-label}
\resizebox{\textwidth}{!}{%
\begin{tabular}{|l|l|l|l|l|}
\hline
\textbf{Class Name}                  & \multicolumn{4}{l|}{com.github.comp354project.service.account.AccountService}                                                                                                                                   \\ \hline
\textbf{Type}                        & \multicolumn{4}{l|}{Class}                                                                                                                                                                                     \\ \hline
\textbf{Inherits}                    & \multicolumn{4}{l|}{N/A}                                                                                                                                                                                        \\ \hline
\textbf{Implements}                  & \multicolumn{4}{l|}{IAccountService}                                                                                                                                                                            \\ \hline
\textbf{Description}                 & \multicolumn{4}{l|}{Class used to request information from the bank database in order to add or delete an account to myMoney application}                                                                       \\ \hline
\multirow{5}{*}{\textbf{Attributes}} & \textbf{Visibility} & \textbf{Data Type}                          & \textbf{Name}        & \textbf{Description}                                                                                                 \\ \cline{2-5}
                                     & private             & Logger                                      & logger               & logger object attribute used to keep track of errors                                                                 \\ \cline{2-5}
                                     & private             & Dao\textless Transaction,Integer\textgreater & transactionDao       & Dao object used to query the database                                                                                \\ \cline{2-5}
                                     & private             & Dao\textless User,Integer\textgreater        & userDao              & Dao object used for quering the database                                                                             \\ \cline{2-5}
                                     & private             & IRemoteAccountService                       & remoteAccountService & attribute used to access database                                                                                    \\ \hline
\multirow{5}{*}{\textbf{Methods}}    & \textbf{Visibility} & \textbf{Name}                               & \textbf{Returns}     & \textbf{Description}                                                                                                 \\ \cline{2-5}
                                     & public              & addAccount                                  & Account              & method to request bank information from the database                                                                 \\ \cline{2-5}
                                     & public              & deleteAccount                               & void                 & method to delete a particular account from myMoney application                                                       \\ \cline{2-5}
                                     & public              & transform                                   & Account              & method to create the appropriate banking info to display for the myMoney app based on the retrieved banking info     \\ \cline{2-5}
                                     & public              & Transaction                                 & transform            & method to create the appropriate transaction info to display for the myMoney app based on the retrieved banking info \\ \hline
\end{tabular}%
}
\end{table}



\begin{table}[H]
\centering
\caption{Class AccountServiceModule}
\label{my-label}
\resizebox{\textwidth}{!}{%
\begin{tabular}{|l|l|l|l|l|}
\hline
\textbf{Class Name}                  & \multicolumn{4}{l|}{com.github.comp354project.service.account.AccountServiceModule}                     \\ \hline
\textbf{Type}                        & \multicolumn{4}{l|}{Class}                                                                                \\ \hline
\textbf{Inherits}                    & \multicolumn{4}{l|}{N/A}                                                                                \\ \hline
\textbf{Implements}                  & \multicolumn{4}{l|}{N/A}                                                                                \\ \hline
\textbf{Description}                 & \multicolumn{4}{l|}{used to return need objects for account and transaction needs}                      \\ \hline
\multirow{2}{*}{\textbf{Attributes}} & \textbf{Visibility} & \textbf{Data Type}        & \textbf{Name}      & \textbf{Description}             \\ \cline{2-5}
                                     & None                & none                      & none               & none                             \\ \hline
\multirow{3}{*}{\textbf{Methods}}    & \textbf{Visibility} & \textbf{Name}             & \textbf{Returns}   & \textbf{Description}             \\ \cline{2-5}
                                     & public              & provideTransactionService & transactionService & return transactionService Object \\ \cline{2-5}
                                     & public              & provideAccountService     & accountService     & returns accountService Object    \\ \hline
\end{tabular}%
}
\end{table}


\begin{table}[H]
\centering
\caption{Interface IAccountService}
\label{my-label}
\resizebox{\textwidth}{!}{%
\begin{tabular}{|l|l|l|l|l|}
\hline
\textbf{Class Name}               & \multicolumn{4}{l|}{com.github.comp354project.service.account.IAccountService}     \\ \hline
\textbf{Type}                     & \multicolumn{4}{l|}{Interface}                                                     \\ \hline
\textbf{Inherits}                 & \multicolumn{4}{l|}{N/A}                                                           \\ \hline
\textbf{Implements}               & \multicolumn{4}{l|}{N/A}                                                           \\ \hline
\textbf{Description}              & \multicolumn{4}{l|}{interface class for adding and deleting an account}            \\ \hline
\textbf{Attributes}               & \textbf{Visibility} & \textbf{Data Type} & \textbf{Name}    & \textbf{Description} \\ \hline
None                              & None                & None               & none             & none                 \\ \hline
\multirow{3}{*}{\textbf{Methods}} & \textbf{Visibility} & \textbf{Name}      & \textbf{Returns} & \textbf{Description} \\ \cline{2-5}
                                  & N/A                 & addAccount         & N/A              & none                 \\ \cline{2-5}
                                  & N/A                 & deleteAccount      & N/A              & none                 \\ \hline
\end{tabular}%
}
\end{table}


\begin{table}[H]
\centering
\caption{Interface ITransactionService}
\label{my-label}
\resizebox{\textwidth}{!}{%
\begin{tabular}{|l|l|l|l|l|}
\hline
\textbf{Class Name}               & \multicolumn{4}{l|}{com.github.comp354project.service.account.ITransactionService}        \\ \hline
\textbf{Type}                     & \multicolumn{4}{l|}{Interface}                                                            \\ \hline
\textbf{Inherits}                 & \multicolumn{4}{l|}{N/A}                                                                  \\ \hline
\textbf{Implements}               & \multicolumn{4}{l|}{N/A}                                                                  \\ \hline
\textbf{Description}              & \multicolumn{4}{l|}{interface class to updating transactions based on categories}         \\ \hline
\textbf{Attributes}               & \textbf{Visibility} & \textbf{Data Type}        & \textbf{Name}    & \textbf{Description} \\ \hline
None                              & None                & None                      & None             & None                 \\ \hline
\multirow{2}{*}{\textbf{Methods}} & \textbf{Visibility} & \textbf{Name}             & \textbf{Returns} & \textbf{Description} \\ \cline{2-5}
                                  & N/A                 & updateTransactionCategory & Transaction      & N/A                  \\ \hline
\end{tabular}%
}
\end{table}


\begin{table}[H]
\centering
\caption{Class Transaction}
\label{my-label}
\resizebox{\textwidth}{!}{%
\begin{tabular}{|l|l|l|l|l|}
\hline
\textbf{Class Name}                  & \multicolumn{4}{l|}{com.github.comp354project.service.account.Transaction}                       \\ \hline
\textbf{Type}                        & \multicolumn{4}{l|}{Class}                                                                         \\ \hline
\textbf{Inherits}                    & \multicolumn{4}{l|}{N/A}                                                                         \\ \hline
\textbf{Implements}                  & \multicolumn{4}{l|}{N/A}                                                                         \\ \hline
\textbf{Description}                 & \multicolumn{4}{l|}{Class used to contain the attributes needed to hold a transaction's details} \\ \hline
\multirow{8}{*}{\textbf{Attributes}} & \textbf{Visibility}  & \textbf{Data Type}  & \textbf{Name}     & \textbf{Description}            \\ \cline{2-5}
                                     & private              & Integer             & date              & date of a transaction           \\ \cline{2-5}
                                     & private              & Double              & amount            & dollar amount of a transaction  \\ \cline{2-5}
                                     & private              & String              & type              & the type of a transaction       \\ \cline{2-5}
                                     & private              & String              & category          & the category of a transaction   \\ \cline{2-5}
                                     & private              & Integer             & sourceID          & ID number                       \\ \cline{2-5}
                                     & private              & Integer             & destinationID     & ID number                       \\ \cline{2-5}
                                     & private              & Account             & account           & name of the account             \\ \hline
\multirow{2}{*}{\textbf{Methods}}    & \textbf{Visibility}  & \textbf{Name}       & \textbf{Returns}  & \textbf{Description}            \\ \cline{2-5}
                                     & None                 & None                & None              & None                            \\ \hline
\end{tabular}%
}
\end{table}


\begin{table}[H]
\centering
\caption{Class TransactionService}
\label{my-label}
\resizebox{\textwidth}{!}{%
\begin{tabular}{|l|l|l|l|l|}
\hline
\textbf{Class Name}                  & \multicolumn{4}{l|}{com.github.comp354project.service.account.TransactionService}                                                              \\ \hline
\textbf{Type}                        & \multicolumn{4}{l|}{Class}                                                                                                                       \\ \hline
\textbf{Inherits}                    & \multicolumn{4}{l|}{N/A}                                                                                                                       \\ \hline
\textbf{Implements}                  & \multicolumn{4}{l|}{ITransactionService}                                                                                                       \\ \hline
\textbf{Description}                 & \multicolumn{4}{l|}{class used to help with transaction changes}                                                                               \\ \hline
\multirow{4}{*}{\textbf{Attributes}} & \textbf{Visibility} & \textbf{Data Type}                          & \textbf{Name}     & \textbf{Description}                                   \\ \cline{2-5}
                                     & private             & Logger                                      & logger            & object used to interact with TransactionService class  \\ \cline{2-5}
                                     & private             & Dao\textless Transaction,Integer\textgreater & transactionDao    & object used to perform methods related to transactions \\ \cline{2-5}
                                     & private             & ICategoryNameValidator                      & categoryValidator & object used to validate if a category is correct       \\ \hline
\multirow{3}{*}{\textbf{Methods}}    & \textbf{Visibility} & \textbf{Name}                               & \textbf{Returns}  & \textbf{Description}                                   \\ \cline{2-5}
                                     & public              & TransactionService                          & N/A               & constructor                                            \\ \cline{2-5}
                                     & public              & updateTransactionCategory                   & Transaction       & used to update a specific transation                   \\ \hline
\end{tabular}%
}
\end{table}

%End of com.github.comp354project.service.account

% Start of com.github.comp354project.service.dao

\begin{table}[H]
\centering
\caption{Class DaoModule}
\label{my-label}
\resizebox{\textwidth}{!}{%
\begin{tabular}{|l|l|l|l|l|}
\hline
\textbf{Class Name}  & \multicolumn{4}{l|}{com.github.comp354project.service.dao.DaoModule}                                                                                                                                                                 \\ \hline
\textbf{Type}        & \multicolumn{4}{l|}{Class}                                                                                                                                                                                                           \\ \hline
\textbf{Inherits}    & \multicolumn{4}{l|}{N/A}                                                                                                                                                                                                             \\ \hline
\textbf{Implements}  & \multicolumn{4}{l|}{N/A}                                                                                                                                                                                                             \\ \hline
\textbf{Description} & \multicolumn{4}{l|}{DAO module to bind interfaces to their interfaces and provide them to the classes that require them}                                                                                                             \\ \hline
\textbf{Attributes}  & \textbf{Visibility} & \textbf{Data Type}                                              & \textbf{Name}                                  & \textbf{Description}                                                                        \\ \hline
                     & private             & Logger                                                          & logger                                         & Logs event information                                                                      \\ \hline
\textbf{Methods}     & \textbf{Visibility} & \textbf{Name}                                                   & \textbf{Returns}                               & \textbf{Description}                                                                        \\ \hline
                     & public              & provideRemoteAccountDao(IConnectionProvider connectionProvider) & Dao\textless RemoteAccount, Integer\textgreater & \begin{tabular}[c]{@{}l@{}}Returns the implementation \\ of a RemoteAccountDao\end{tabular} \\ \hline
                     & public              & provideAccountDao(IConnectionProvider connectionProvider)       & Dao\textless Account, Integer\textgreater       & \begin{tabular}[c]{@{}l@{}}Returns the implementation\\ of an AccountDao\end{tabular}       \\ \hline
                     & public              & provideTransactionDao(IConnectionProvider connectionProvider)   & Dao\textless Transaction, Integer\textgreater   & \begin{tabular}[c]{@{}l@{}}Returns the implementation\\ of a TransactionDao\end{tabular}    \\ \hline
                     & public              & provideUserDao(IConnectionProvider connectionProvider)          & Dao\textless User, Integer\textgreater          & \begin{tabular}[c]{@{}l@{}}Returns the implementation\\ of a UserDao\end{tabular}           \\ \hline
\end{tabular}%
}
\end{table}

%end of com.github.comp354project.service.dao

% Start of com.github.comp354project.service.sqlite

\begin{table}[H]
\centering
\caption{Class ConnectionModule}
\label{my-label}
\resizebox{\textwidth}{!}{%
\begin{tabular}{|l|l|l|l|l|}
\hline
\textbf{Class Name}  & \multicolumn{4}{l|}{com.github.comp354project.service.sqlite.ConnectionModule}                                                                                                                      \\ \hline
\textbf{Type}        & \multicolumn{4}{l|}{Class}                                                                                                                                                                          \\ \hline
\textbf{Inherits}    & \multicolumn{4}{l|}{N/A}                                                                                                                                                                            \\ \hline
\textbf{Implements}  & \multicolumn{4}{l|}{N/A}                                                                                                                                                                            \\ \hline
\textbf{Description} & \multicolumn{4}{l|}{Module that creates a connection to the database}                                                                                                                               \\ \hline
\textbf{Attributes}  & \textbf{Visibility} & \textbf{Data Type}                                       & \textbf{Name}       & \textbf{Description}                                                                         \\ \hline
None                 &                     &                                                          &                     &                                                                                              \\ \hline
\textbf{Methods}     & \textbf{Visibility} & \textbf{Name}                                            & \textbf{Returns}    & \textbf{Description}                                                                         \\ \hline
                     & protected           & provideConnection(ConnectionProvider connectionProvider) & IConnectionProvider & \begin{tabular}[c]{@{}l@{}}Returns the implementation\\ of a ConnectionProvider\end{tabular} \\ \hline
\end{tabular}%
}
\end{table}

\begin{table}[H]
\centering
\caption{Class ConnectionProvider}
\label{my-label}
\resizebox{\textwidth}{!}{%
\begin{tabular}{|l|l|l|l|l|}
\hline
\textbf{Class Name}  & \multicolumn{4}{l|}{com.github.comp354project.service.sqlite.ConnectionProvider}                                                                    \\ \hline
\textbf{Type}        & \multicolumn{4}{l|}{Class}                                                                                                                          \\ \hline
\textbf{Inherits}    & \multicolumn{4}{l|}{N/A}                                                                                                                            \\ \hline
\textbf{Implements}  & \multicolumn{4}{l|}{IConnectionProvider}                                                                                                            \\ \hline
\textbf{Description} & \multicolumn{4}{l|}{Instatiates a connection to an SQLite database}                                                                                 \\ \hline
\textbf{Attributes}  & \textbf{Visibility} & \textbf{Data Type}    & \textbf{Name}        & \textbf{Description}                                                           \\ \hline
                     & private             & Logger                & logger               & Logs events                                                                    \\ \hline
\textbf{Methods}     & \textbf{Visibility} & \textbf{Name}         & \textbf{Returns}     & \textbf{Description}                                                           \\ \hline
                     & public              & ConnectionProvider()  & ConnectionProvider   & Constructs the class                                                           \\ \hline
                     & public              & getConnectionSource() & JdbcConnectionSource & \begin{tabular}[c]{@{}l@{}}Returns a database connection\\ source\end{tabular} \\ \hline
\end{tabular}%
}
\end{table}

\begin{table}[H]
\centering
\caption{Interface IConnectionProvider}
\label{my-label}
\resizebox{\textwidth}{!}{%
\begin{tabular}{|l|l|l|l|l|}
\hline
\textbf{Class Name}  & \multicolumn{4}{l|}{com.github.comp354project.service.sqlite.IConnectionProvider}                                                                   \\ \hline
\textbf{Type}        & \multicolumn{4}{l|}{Interface}                                                                                                                      \\ \hline
\textbf{Inherits}    & \multicolumn{4}{l|}{N/A}                                                                                                                            \\ \hline
\textbf{Implements}  & \multicolumn{4}{l|}{N/A}                                                                                                                            \\ \hline
\textbf{Description} & \multicolumn{4}{l|}{Instatiates a connection to a database}                                                                                         \\ \hline
\textbf{Attributes}  & \textbf{Visibility} & \textbf{Data Type}    & \textbf{Name}        & \textbf{Description}                                                           \\ \hline
None                 &                     &                       &                      &                                                                                \\ \hline
\textbf{Methods}     & \textbf{Visibility} & \textbf{Name}         & \textbf{Returns}     & \textbf{Description}                                                           \\ \hline
                     & public              & getConnectionSource() & JdbcConnectionSource & \begin{tabular}[c]{@{}l@{}}Returns a database connection\\ source\end{tabular} \\ \hline
\end{tabular}%
}
\end{table}

% End of com.github.comp354project.service.sqlite
%Start of com.github.comp354project.account.remote

\begin{table}[]
\centering
\caption{Class GetRemoteAccountRequest}
\label{my-label}
\resizebox{\textwidth}{!}{%
\begin{tabular}{|l|l|l|l|l|l|}
\hline
\textbf{Class Name}  & \multicolumn{5}{l|}{com.github.comp354project.service.account.remote.GetRemoteAccountRequest}                                                                   \\ \hline
\textbf{Type}        & \multicolumn{5}{l|}{Public}                                                                                    \\ \hline
\textbf{Inherits}    & \multicolumn{5}{l|}{N/A}                                                                                       \\ \hline
\textbf{Implements}  & \multicolumn{5}{l|}{N/A}                                                                                       \\ \hline
\textbf{Description} & \multicolumn{5}{l|}{Retrieve the remote account request}                                                       \\ \hline
\textbf{Attributes}  & \textbf{Visibility} & \textbf{Data Type} & \textbf{Name}    & \multicolumn{2}{l|}{\textbf{Description}}        \\ \hline
                     & Private             & Integer            & accountID        & \multicolumn{2}{l|}{Identification of an account} \\ \hline
\textbf{Methods}     & \textbf{Visibility} & \textbf{Name}      & \textbf{Returns} & \textbf{Description}       & \textbf{Throws}     \\ \hline
None                 &                     &                    &                  &                            &                     \\ \hline
\end{tabular}%
}
\end{table}


\begin{table}[]
\centering
\caption{Class GetRemoteAccountResponse}
\label{my-label}
\resizebox{\textwidth}{!}{%
\begin{tabular}{|l|l|l|l|l|l|}
\hline
\textbf{Class Name}  & \multicolumn{5}{l|}{com.github.comp354project.service.account.remote.GetRemoteAccountResponse}                                                           \\ \hline
\textbf{Type}        & \multicolumn{5}{l|}{Public}                                                                             \\ \hline
\textbf{Inherits}    & \multicolumn{5}{l|}{N/A}                                                                                \\ \hline
\textbf{Implements}  & \multicolumn{5}{l|}{N/A}                                                                                \\ \hline
\textbf{Description} & \multicolumn{5}{l|}{Retrieve the response for the remote account request}                               \\ \hline
\textbf{Attributes}  & \textbf{Visibility} & \textbf{Data Type} & \textbf{Name}    & \multicolumn{2}{l|}{\textbf{Description}} \\ \hline
                     & Private             & RemoteAccount      & account          & \multicolumn{2}{l|}{The remote account}   \\ \hline
\textbf{Methods}     & \textbf{Visibility} & \textbf{Name}      & \textbf{Returns} & \textbf{Description}   & \textbf{Throws}  \\ \hline
None                 &                     &                    &                  &                        &                  \\ \hline
\end{tabular}%
}
\end{table}




\begin{table}[]
\centering
\caption{Interface IRemoteAccountService}
\label{my-label}
\resizebox{\textwidth}{!}{%
\begin{tabular}{|l|l|l|l|l|l|}
\hline
\textbf{Name}        & \multicolumn{5}{l|}{com.github.comp354project.service.account.remote.IRemoteAccountService}                                  \\ \hline
\textbf{Type}        & \multicolumn{5}{l|}{Public}                                                                                                \\ \hline
\textbf{Inherits}    & \multicolumn{5}{l|}{N/A}                                                                                                   \\ \hline
\textbf{Implements}  & \multicolumn{5}{l|}{N/A}                                                                                                   \\ \hline
\textbf{Description} & \multicolumn{5}{l|}{The interface for the remote account service (request and response)}                                   \\ \hline
\textbf{Attributes}  & \textbf{Visibility}     & \textbf{Data Type}        & \textbf{Name}          & \multicolumn{2}{l|}{\textbf{Description}}   \\ \hline
None                 &                         &                           &                        & \multicolumn{2}{l|}{}                       \\ \hline
\textbf{Methods}     & \textbf{Visibility}     & \textbf{Name}             & \textbf{Throws}        & \multicolumn{2}{l|}{\textbf{Description}}   \\ \hline
                     & Public                  & IRemoteAccountService     & ValidationException    & \multicolumn{2}{l|}{Remote account service} \\ \hline
\end{tabular}%
}
\end{table}



\begin{table}[]
\centering
\caption{Class RemoteAccount}
\label{my-label}
\resizebox{\textwidth}{!}{%
\begin{tabular}{|l|l|l|l|l|l|}
\hline
\textbf{Class Name}  & \multicolumn{5}{l|}{com.github.comp354project.service.account.remote.RemoteAccount}                            \\ \hline
\textbf{Type}        & \multicolumn{5}{l|}{Public}                                                                                    \\ \hline
\textbf{Inherits}    & \multicolumn{5}{l|}{N/A}                                                                                       \\ \hline
\textbf{Implements}  & \multicolumn{5}{l|}{N/A}                                                                                       \\ \hline
\textbf{Description} & \multicolumn{5}{l|}{The remote account with details}                                                           \\ \hline
\textbf{Attributes}  & \textbf{Visibility} & \textbf{Data Type} & \textbf{Name}    & \multicolumn{2}{l|}{\textbf{Description}}        \\ \hline
\multirow{5}{*}{}    & Private             & Integer            & ID               & \multicolumn{2}{l|}{ID of the account}           \\ \cline{2-6}
                     & Private             & String             & bankName         & \multicolumn{2}{l|}{Name of the bank}            \\ \cline{2-6}
                     & Private             & String             & Type             & \multicolumn{2}{l|}{Type of the account}         \\ \cline{2-6}
                     & Private             & Double             & balance          & \multicolumn{2}{l|}{Balance of the account}      \\ \cline{2-6}
                     & Private             & ForeignCollection  & transactions     & \multicolumn{2}{l|}{Transactions of the account} \\ \hline
\textbf{Methods}     & \textbf{Visibility} & \textbf{Name}      & \textbf{Returns} & \textbf{Description}       & \textbf{Throws}     \\ \hline
None                 &                     &                    &                  &                            &                     \\ \hline
\end{tabular}%
}
\end{table}



\begin{table}[]
\centering
\caption{Class RemoteAccountModule}
\label{my-label}
\resizebox{\textwidth}{!}{%
\begin{tabular}{|l|l|l|l|l|}
\hline
\textbf{Class Name}  & \multicolumn{4}{l|}{com.github.comp354project.service.account.remote.RemoteAccountModule}                                                                                    \\ \hline
\textbf{Type}        & \multicolumn{4}{l|}{Public}                                                                                                                                                  \\ \hline
\textbf{Inherits}    & \multicolumn{4}{l|}{N/A}                                                                                                                                                     \\ \hline
\textbf{Implements}  & \multicolumn{4}{l|}{N/A}                                                                                                                                                     \\ \hline
\textbf{Description} & \multicolumn{4}{l|}{The module for remote account class}                                                                                                                     \\ \hline
\textbf{Attributes}  & \textbf{Visibility}      & \textbf{Data Type}                           & \textbf{Name}                         & \textbf{Description}                                       \\ \hline
None                 &                          &                                              &                                       &                                                            \\ \hline
\textbf{Methods}     & \textbf{Visibility}      & \textbf{Name}                                & \textbf{Returns}                      & \textbf{Description}                                       \\ \hline
\multirow{2}{*}{}    & \multirow{2}{*}{Default} & \multirow{2}{*}{provideRemoteAccountService} & \multirow{2}{*}{remoteAccountService} & \multirow{2}{*}{Module provide the remote account service} \\
                     &                          &                                              &                                       &                                                            \\ \hline
\end{tabular}%
}
\end{table}




\begin{table}[]
\centering
\caption{Class RemoteAccountService}
\label{my-label}
\resizebox{\textwidth}{!}{%
\begin{tabular}{|l|l|l|l|l|l|}
\hline
\textbf{Class Name}                  & \multicolumn{5}{l|}{com.github.comp354project.service.account.remote.RemoteAccountService}                                                                                                                                                                                                                                                                                                                                              \\ \hline
\textbf{Type}                        & \multicolumn{5}{l|}{Public}                                                                                                                                                                                                                                                                                                                                                            \\ \hline
\textbf{Inherits}                    & \multicolumn{5}{l|}{N/A}                                                                                                                                                                                                                                                                                                                                                               \\ \hline
\textbf{Implements}                  & \multicolumn{5}{l|}{IRemoteAccountService}                                                                                                                                                                                                                                                                                                                                             \\ \hline
\textbf{Description}                 & \multicolumn{5}{l|}{The services that the remote account can provide}                                                                                                                                                                                                                                                                                                                  \\ \hline
\multirow{2}{*}{\textbf{Attributes}} & \textbf{Visibility}         & \textbf{Data Type}                                                                                            & \textbf{Name}                                                                       & \multicolumn{2}{l|}{\textbf{Description}}                                                                                                          \\ \cline{2-6}
                                     & Private                     & Logger                                                                                                        & logger                                                                              & \multicolumn{2}{l|}{\begin{tabular}[c]{@{}l@{}}Gets the log \\ of the Remote\\ AccountService.class\end{tabular}}                                  \\ \hline
                                     & Private                     & \begin{tabular}[c]{@{}l@{}}Dao \textless Remote\\ Account, Integer \textgreater\end{tabular}                  & \begin{tabular}[c]{@{}l@{}}Remote\\ AccountDao\end{tabular}                         & \multicolumn{2}{l|}{RemoteAccountDoa}                                                                                                              \\ \hline
\textbf{Methods}                     & \textbf{Visibility}         & \textbf{Name}                                                                                                 & \textbf{Throws}                                                                     & \multicolumn{2}{l|}{\textbf{Description}}                                                                                                          \\ \hline
                                     & \multicolumn{1}{c|}{Public} & \multicolumn{1}{c|}{RemoteAccountService(Dao  \textless RemoteAccount,Integer \textgreater remoteAccountDao)} & \multicolumn{1}{c|}{N/A}                                                            & \multicolumn{2}{l|}{\begin{tabular}[c]{@{}l@{}}The constructor \\ class for Remote\\ AccountService\end{tabular}}                                  \\ \hline
                                     & Public                      & getAccount(GetRemoteAccountRequest request)                                                                   & \multicolumn{1}{c|}{\begin{tabular}[c]{@{}c@{}}Validation\\ Exception\end{tabular}} & \multicolumn{2}{l|}{\begin{tabular}[c]{@{}l@{}}Return the account \\ information if \\ there is a request \\ for it and if it exists\end{tabular}} \\ \hline
\end{tabular}%
}
\end{table}



\begin{table}[]
\centering
\caption{Class RemoteTransaction}
\label{my-label}
\resizebox{\textwidth}{!}{%
\begin{tabular}{|l|l|l|l|l|}
\hline
\textbf{Class Name}                  & \multicolumn{4}{l|}{com.github.comp354project.service.account.remote.RemoteTransaction}                                                                                                                                           \\ \hline
\textbf{Type}                        & \multicolumn{4}{l|}{Public}                                                                                                                                                      \\ \hline
\textbf{Inherits}                    & \multicolumn{4}{l|}{N/A}                                                                                                                                                         \\ \hline
\textbf{Implements}                  & \multicolumn{4}{l|}{N/A}                                                                                                                                                         \\ \hline
\textbf{Description}                 & \multicolumn{4}{l|}{The remote transaction class}                                                                                                                                \\ \hline
\multirow{3}{*}{\textbf{Attributes}} & \textbf{Visibility} & \textbf{Data Type} & \textbf{Name}    & \textbf{Description}                                                                                               \\ \cline{2-5}
                                     & Private             & Integer            & ID               & \begin{tabular}[c]{@{}l@{}}Identification of the \\ remote transaction\end{tabular}                                \\ \cline{2-5}
                                     & Private             & Integer            & date             & Date of the transaction                                                                                            \\ \hline
                                     & Private             & Double             & amount           & Amount of money transitioned                                                                                       \\ \hline
                                     & Private             & String             & type             & Type of transaction                                                                                                \\ \hline
                                     & Private             & Integer            & SourceID         & \begin{tabular}[c]{@{}l@{}}Identification of the source \\ where the money was originally\\ resided\end{tabular}   \\ \hline
                                     & Private             & Integer            & destinationID    & \begin{tabular}[c]{@{}l@{}}Identification of the destination\\ where the money will be\\ transitioned\end{tabular} \\ \hline
                                     & Private             & Remote             & account          & The main account of the user                                                                                       \\ \hline
\textbf{Methods}                     & \textbf{Visibility} & \textbf{Name}      & \textbf{Returns} & \textbf{Description}                                                                                               \\ \hline
None                                 &                     &                    &                  &                                                                                                                    \\ \hline
\end{tabular}%
}
\end{table}
%end of Start of com.github.comp354project.account.remote

\clearpage

\subsection{Glossary of Domain Concepts} \label{glossary}
\begin{table}[H]
\caption{Glossary of Domain Concepts}
\begin{center}
\scalebox{0.7}{
\begin{tabular}{|l|p{0.8\linewidth}|}
\hline
Expression &  Definition \\
\hline
User & The person that is using the application and the main provider of requests to the system. \\
\hline
User Account & A data object containing user information. It also contains the various bank accounts that a user may have linked to the system. \\
\hline
Bank Account & A data object containing transactions linked with a specific bank account in a bank institution. One user account may have more than one bank accounts.\\
\hline
Transaction & Any kind of money exchange associated with a bank account.\\
\hline
Transfer & A type of transaction that occurs between two parties.\\
\hline
Deposit & A type of transaction where the owner puts money in his own bank account.\\
\hline
Withdrawal & A type of transaction where the owner of the bank account removes money from his balance.\\
\hline
Database & A local or online container which holds data in an organized, efficient manner. \\
\hline
Server & a computer that is accessible on a network, on which a database and/or system may be hosted. The bank institutions' databases will be hosted on here. \\
\hline
Object-Oriented Programming & A programming paradigm which separates entities into objects, and uses the concept of inheritance of properties, polymorphism of objects, encapsulation of objects. We use this paradigm for its maintainability and structural benefits.\\
\hline
MVC - Model-View-Controller Architecture & An architectural pattern which strictly separates components into the model (manages the data and logic), the view (output of the model), and the controller (handling input and passing it to the model or view).\\
\hline
Interface & A component of a system by which other entities (be it humans or other systems) may engage in an exchange of data with the system in question. \\
\hline
API - Application Programming Interface & A protocol or set of functions which serve as a method of communication to a software system. It is a type of interface, and the one by which our system will communicate with the banking institutions' databases. \\
\hline
DAO - Data access object & An object that provides an abstract interface to some type of database or other persistence mechanism.\\
\hline

\end{tabular}}
\end{center}
\end{table}

\clearpage

\subsection{Subsystem X}

\subsubsection{Detailed Design Diagram}

UML class diagram depicting the internal structure of the subsystem,
accompanied by a paragraph of text describing the rationale of this design.

*Note: The above is a description of what to provide. Need to edit into our own




\subsubsection{Units Description}

List each class in this subsystem and write a short description of its purpose,
as well as notes or reminders useful for the programmers who will implement them.
List all attributes and functions of the class.

*Note: The above is a description of what to provide. Need to edit into our own

\section{Dynamic Design Scenarios} \label{Dynamic Models}

Describe some (at least two) important execution scenarios of the system using UML sequence diagrams.
These scenarios must demonstrate how the various subsystems and units are interacting to achieve a system-level service.
Units and subsystems depicted here must be compatible with the descriptions provided in
section \ref{sec:arch} and \ref{sec:detail}.

*Note: The above is a description of what to provide. Need to edit into our own


\subsection{Dynamic Models of System Interface}

We have chosen 3 major functionalities of the system (also known as use cases) in order to portray the interactions between the classes of the system. By using a sequence diagram, this will display the dynamics visually by showcasing the sequences of method calls when a particular use case begins functioning.

\clearpage

\subsubsection{Use Case 1: Create User Account}

The following scenario describes the actions that occur when the user clicks on the sign up button

\begin{figure}[H]
\includegraphics[width=\linewidth]{case1SequenceDiagram.png}
\caption{Use case 1 Sequence Diagram}
\label{fig:use-case-1-sequence-diagram}
\end{figure}

\clearpage

\subsubsection{Use Case 3: Add Bank Account to a User Account}

The following scenario describes the actions that occur when a user clicks the add button in the account list view.

\begin{figure}[H]
\includegraphics[width=\linewidth]{usecase3_sequence_diagram.png}
\caption{Use case 3 Sequence Diagram}
\label{fig:use-case-3-sequence-diagram}
\end{figure}

\clearpage

\subsubsection{Use Case 5: View Transactions for Specific Bank Account}

The following scenario describes the actions that occur when the user clicks the button<view transactions> for a specific bank account.

\begin{figure}[H]
\includegraphics[width=\linewidth]{usecase5.png}
\caption{Use case 5 Sequence Diagram}
\label{fig:use-case-5-sequence-diagram}
\end{figure}

\clearpage

\clearpage

\subsubsection{Use Case 6: View All Transactions from all Bank Accounts}

The following scenario describes the actions that occur when the user click the button "view all transactions" for viewing all transactions from all bank accounts.

\begin{figure}[H]
\includegraphics[width=\linewidth]{Use_Case_6_Sequence_Diagram.png}
\caption{UseCase 6 Sequence Diagram}
\label{fig:use-case-6-sequence-diagram}
\end{figure}

\clearpage

\section{Reference}

\begin{itemize}
\item User information: As our user and use-cases was based on feedback provided by our developers, our references lie mainly within our own team.
\item Craig Larman - Applying UML and Patterns
\item Greg Butler's course COMP 354 content
\item \href{http://web.mit.edu/ssit/cis/CISRequirements.html}{\textcolor{blue}{MIT Curricular Information System
Software Requirements Document}}
\item \href{https://resources.sei.cmu.edu/asset_files/TechnicalReport/2005_005_001_14621.pdf}{\textcolor{blue}{Carnegie Mellon Business Goals}}
\item \href{http://www.oracle.com/technetwork/testcontent/gettingstartedwithusecasemodeling-133857.pdf}{\textcolor{blue}{Use-Case: Oracle }}
\item \href{https://github.com/google/dagger}{\textcolor{blue}{Google Dagger Github}}

\end{itemize}

\end{document}
\resizebox{\textwidth}{!}{%
\begin{tabular}{|l|l|l|l|l|}
\hline
\textbf{Class Name}  & \multicolumn{4}{l|}{com.github.comp354project.service.sqlite.IConnectionProvider}                                                                   \\ \hline
\textbf{Type}        & \multicolumn{4}{l|}{Interface}                                                                                                                      \\ \hline
\textbf{Inherits}    & \multicolumn{4}{l|}{N/A}                                                                                                                            \\ \hline
\textbf{Implements}  & \multicolumn{4}{l|}{N/A}                                                                                                                            \\ \hline
\textbf{Description} & \multicolumn{4}{l|}{Instatiates a connection to a database}                                                                                         \\ \hline
\textbf{Attributes}  & \textbf{Visibility} & \textbf{Data Type}    & \textbf{Name}        & \textbf{Description}                                                           \\ \hline
None                 &                     &                       &                      &                                                                                \\ \hline
\textbf{Methods}     & \textbf{Visibility} & \textbf{Name}         & \textbf{Returns}     & \textbf{Description}                                                           \\ \hline
                     & public              & getConnectionSource() & JdbcConnectionSource & \begin{tabular}[c]{@{}l@{}}Returns a database connection\\ source\end{tabular} \\ \hline
\end{tabular}%
}
\end{table}

% End of com.github.comp354project.service.sqlite
%Start of com.github.comp354project.account.remote

\begin{table}[]
\centering
\caption{Class GetRemoteAccountRequest}
\label{my-label}
\resizebox{\textwidth}{!}{%
\begin{tabular}{|l|l|l|l|l|l|}
\hline
\textbf{Class Name}  & \multicolumn{5}{l|}{com.github.comp354project.service.account.remote.GetRemoteAccountRequest}                                                                   \\ \hline
\textbf{Type}        & \multicolumn{5}{l|}{Public}                                                                                    \\ \hline
\textbf{Inherits}    & \multicolumn{5}{l|}{N/A}                                                                                       \\ \hline
\textbf{Implements}  & \multicolumn{5}{l|}{N/A}                                                                                       \\ \hline
\textbf{Description} & \multicolumn{5}{l|}{Retrieve the remote account request}                                                       \\ \hline
\textbf{Attributes}  & \textbf{Visibility} & \textbf{Data Type} & \textbf{Name}    & \multicolumn{2}{l|}{\textbf{Description}}        \\ \hline
                     & Private             & Integer            & accountID        & \multicolumn{2}{l|}{Identification of an account} \\ \hline
\textbf{Methods}     & \textbf{Visibility} & \textbf{Name}      & \textbf{Returns} & \textbf{Description}       & \textbf{Throws}     \\ \hline
None                 &                     &                    &                  &                            &                     \\ \hline
\end{tabular}%
}
\end{table}


\begin{table}[]
\centering
\caption{Class GetRemoteAccountResponse}
\label{my-label}
\resizebox{\textwidth}{!}{%
\begin{tabular}{|l|l|l|l|l|l|}
\hline
\textbf{Class Name}  & \multicolumn{5}{l|}{com.github.comp354project.service.account.remote.GetRemoteAccountResponse}                                                           \\ \hline
\textbf{Type}        & \multicolumn{5}{l|}{Public}                                                                             \\ \hline
\textbf{Inherits}    & \multicolumn{5}{l|}{N/A}                                                                                \\ \hline
\textbf{Implements}  & \multicolumn{5}{l|}{N/A}                                                                                \\ \hline
\textbf{Description} & \multicolumn{5}{l|}{Retrieve the response for the remote account request}                               \\ \hline
\textbf{Attributes}  & \textbf{Visibility} & \textbf{Data Type} & \textbf{Name}    & \multicolumn{2}{l|}{\textbf{Description}} \\ \hline
                     & Private             & RemoteAccount      & account          & \multicolumn{2}{l|}{The remote account}   \\ \hline
\textbf{Methods}     & \textbf{Visibility} & \textbf{Name}      & \textbf{Returns} & \textbf{Description}   & \textbf{Throws}  \\ \hline
None                 &                     &                    &                  &                        &                  \\ \hline
\end{tabular}%
}
\end{table}




\begin{table}[]
\centering
\caption{Interface IRemoteAccountService}
\label{my-label}
\resizebox{\textwidth}{!}{%
\begin{tabular}{|l|l|l|l|l|l|}
\hline
\textbf{Name}        & \multicolumn{5}{l|}{com.github.comp354project.service.account.remote.IRemoteAccountService}                                  \\ \hline
\textbf{Type}        & \multicolumn{5}{l|}{Public}                                                                                                \\ \hline
\textbf{Inherits}    & \multicolumn{5}{l|}{N/A}                                                                                                   \\ \hline
\textbf{Implements}  & \multicolumn{5}{l|}{N/A}                                                                                                   \\ \hline
\textbf{Description} & \multicolumn{5}{l|}{The interface for the remote account service (request and response)}                                   \\ \hline
\textbf{Attributes}  & \textbf{Visibility}     & \textbf{Data Type}        & \textbf{Name}          & \multicolumn{2}{l|}{\textbf{Description}}   \\ \hline
None                 &                         &                           &                        & \multicolumn{2}{l|}{}                       \\ \hline
\textbf{Methods}     & \textbf{Visibility}     & \textbf{Name}             & \textbf{Throws}        & \multicolumn{2}{l|}{\textbf{Description}}   \\ \hline
                     & Public                  & IRemoteAccountService     & ValidationException    & \multicolumn{2}{l|}{Remote account service} \\ \hline
\end{tabular}%
}
\end{table}



\begin{table}[]
\centering
\caption{Class RemoteAccount}
\label{my-label}
\resizebox{\textwidth}{!}{%
\begin{tabular}{|l|l|l|l|l|l|}
\hline
\textbf{Class Name}  & \multicolumn{5}{l|}{com.github.comp354project.service.account.remote.RemoteAccount}                            \\ \hline
\textbf{Type}        & \multicolumn{5}{l|}{Public}                                                                                    \\ \hline
\textbf{Inherits}    & \multicolumn{5}{l|}{N/A}                                                                                       \\ \hline
\textbf{Implements}  & \multicolumn{5}{l|}{N/A}                                                                                       \\ \hline
\textbf{Description} & \multicolumn{5}{l|}{The remote account with details}                                                           \\ \hline
\textbf{Attributes}  & \textbf{Visibility} & \textbf{Data Type} & \textbf{Name}    & \multicolumn{2}{l|}{\textbf{Description}}        \\ \hline
\multirow{5}{*}{}    & Private             & Integer            & ID               & \multicolumn{2}{l|}{ID of the account}           \\ \cline{2-6}
                     & Private             & String             & bankName         & \multicolumn{2}{l|}{Name of the bank}            \\ \cline{2-6}
                     & Private             & String             & Type             & \multicolumn{2}{l|}{Type of the account}         \\ \cline{2-6}
                     & Private             & Double             & balance          & \multicolumn{2}{l|}{Balance of the account}      \\ \cline{2-6}
                     & Private             & ForeignCollection  & transactions     & \multicolumn{2}{l|}{Transactions of the account} \\ \hline
\textbf{Methods}     & \textbf{Visibility} & \textbf{Name}      & \textbf{Returns} & \textbf{Description}       & \textbf{Throws}     \\ \hline
None                 &                     &                    &                  &                            &                     \\ \hline
\end{tabular}%
}
\end{table}



\begin{table}[]
\centering
\caption{Class RemoteAccountModule}
\label{my-label}
\resizebox{\textwidth}{!}{%
\begin{tabular}{|l|l|l|l|l|}
\hline
\textbf{Class Name}  & \multicolumn{4}{l|}{com.github.comp354project.service.account.remote.RemoteAccountModule}                                                                                    \\ \hline
\textbf{Type}        & \multicolumn{4}{l|}{Public}                                                                                                                                                  \\ \hline
\textbf{Inherits}    & \multicolumn{4}{l|}{N/A}                                                                                                                                                     \\ \hline
\textbf{Implements}  & \multicolumn{4}{l|}{N/A}                                                                                                                                                     \\ \hline
\textbf{Description} & \multicolumn{4}{l|}{The module for remote account class}                                                                                                                     \\ \hline
\textbf{Attributes}  & \textbf{Visibility}      & \textbf{Data Type}                           & \textbf{Name}                         & \textbf{Description}                                       \\ \hline
None                 &                          &                                              &                                       &                                                            \\ \hline
\textbf{Methods}     & \textbf{Visibility}      & \textbf{Name}                                & \textbf{Returns}                      & \textbf{Description}                                       \\ \hline
\multirow{2}{*}{}    & \multirow{2}{*}{Default} & \multirow{2}{*}{provideRemoteAccountService} & \multirow{2}{*}{remoteAccountService} & \multirow{2}{*}{Module provide the remote account service} \\
                     &                          &                                              &                                       &                                                            \\ \hline
\end{tabular}%
}
\end{table}




\begin{table}[]
\centering
\caption{Class RemoteAccountService}
\label{my-label}
\resizebox{\textwidth}{!}{%
\begin{tabular}{|l|l|l|l|l|l|}
\hline
\textbf{Class Name}                  & \multicolumn{5}{l|}{com.github.comp354project.service.account.remote.RemoteAccountService}                                                                                                                                                                                                                                                                                                                                              \\ \hline
\textbf{Type}                        & \multicolumn{5}{l|}{Public}                                                                                                                                                                                                                                                                                                                                                            \\ \hline
\textbf{Inherits}                    & \multicolumn{5}{l|}{N/A}                                                                                                                                                                                                                                                                                                                                                               \\ \hline
\textbf{Implements}                  & \multicolumn{5}{l|}{IRemoteAccountService}                                                                                                                                                                                                                                                                                                                                             \\ \hline
\textbf{Description}                 & \multicolumn{5}{l|}{The services that the remote account can provide}                                                                                                                                                                                                                                                                                                                  \\ \hline
\multirow{2}{*}{\textbf{Attributes}} & \textbf{Visibility}         & \textbf{Data Type}                                                                                            & \textbf{Name}                                                                       & \multicolumn{2}{l|}{\textbf{Description}}                                                                                                          \\ \cline{2-6}
                                     & Private                     & Logger                                                                                                        & logger                                                                              & \multicolumn{2}{l|}{\begin{tabular}[c]{@{}l@{}}Gets the log \\ of the Remote\\ AccountService.class\end{tabular}}                                  \\ \hline
                                     & Private                     & \begin{tabular}[c]{@{}l@{}}Dao \textless Remote\\ Account, Integer \textgreater\end{tabular}                  & \begin{tabular}[c]{@{}l@{}}Remote\\ AccountDao\end{tabular}                         & \multicolumn{2}{l|}{RemoteAccountDoa}                                                                                                              \\ \hline
\textbf{Methods}                     & \textbf{Visibility}         & \textbf{Name}                                                                                                 & \textbf{Throws}                                                                     & \multicolumn{2}{l|}{\textbf{Description}}                                                                                                          \\ \hline
                                     & \multicolumn{1}{c|}{Public} & \multicolumn{1}{c|}{RemoteAccountService(Dao  \textless RemoteAccount,Integer \textgreater remoteAccountDao)} & \multicolumn{1}{c|}{N/A}                                                            & \multicolumn{2}{l|}{\begin{tabular}[c]{@{}l@{}}The constructor \\ class for Remote\\ AccountService\end{tabular}}                                  \\ \hline
                                     & Public                      & getAccount(GetRemoteAccountRequest request)                                                                   & \multicolumn{1}{c|}{\begin{tabular}[c]{@{}c@{}}Validation\\ Exception\end{tabular}} & \multicolumn{2}{l|}{\begin{tabular}[c]{@{}l@{}}Return the account \\ information if \\ there is a request \\ for it and if it exists\end{tabular}} \\ \hline
\end{tabular}%
}
\end{table}



\begin{table}[]
\centering
\caption{Class RemoteTransaction}
\label{my-label}
\resizebox{\textwidth}{!}{%
\begin{tabular}{|l|l|l|l|l|}
\hline
\textbf{Class Name}                  & \multicolumn{4}{l|}{com.github.comp354project.service.account.remote.RemoteTransaction}                                                                                                                                           \\ \hline
\textbf{Type}                        & \multicolumn{4}{l|}{Public}                                                                                                                                                      \\ \hline
\textbf{Inherits}                    & \multicolumn{4}{l|}{N/A}                                                                                                                                                         \\ \hline
\textbf{Implements}                  & \multicolumn{4}{l|}{N/A}                                                                                                                                                         \\ \hline
\textbf{Description}                 & \multicolumn{4}{l|}{The remote transaction class}                                                                                                                                \\ \hline
\multirow{3}{*}{\textbf{Attributes}} & \textbf{Visibility} & \textbf{Data Type} & \textbf{Name}    & \textbf{Description}                                                                                               \\ \cline{2-5}
                                     & Private             & Integer            & ID               & \begin{tabular}[c]{@{}l@{}}Identification of the \\ remote transaction\end{tabular}                                \\ \cline{2-5}
                                     & Private             & Integer            & date             & Date of the transaction                                                                                            \\ \hline
                                     & Private             & Double             & amount           & Amount of money transitioned                                                                                       \\ \hline
                                     & Private             & String             & type             & Type of transaction                                                                                                \\ \hline
                                     & Private             & Integer            & SourceID         & \begin{tabular}[c]{@{}l@{}}Identification of the source \\ where the money was originally\\ resided\end{tabular}   \\ \hline
                                     & Private             & Integer            & destinationID    & \begin{tabular}[c]{@{}l@{}}Identification of the destination\\ where the money will be\\ transitioned\end{tabular} \\ \hline
                                     & Private             & Remote             & account          & The main account of the user                                                                                       \\ \hline
\textbf{Methods}                     & \textbf{Visibility} & \textbf{Name}      & \textbf{Returns} & \textbf{Description}                                                                                               \\ \hline
None                                 &                     &                    &                  &                                                                                                                    \\ \hline
\end{tabular}%
}
\end{table}
%end of Start of com.github.comp354project.account.remote

\clearpage

\subsection{Glossary of Domain Concepts} \label{glossary}
\begin{table}[H]
\caption{Glossary of Domain Concepts}
\begin{center}
\scalebox{0.7}{
\begin{tabular}{|l|p{0.8\linewidth}|}
\hline
Expression &  Definition \\
\hline
User & The person that is using the application and the main provider of requests to the system. \\
\hline
User Account & A data object containing user information. It also contains the various bank accounts that a user may have linked to the system. \\
\hline
Bank Account & A data object containing transactions linked with a specific bank account in a bank institution. One user account may have more than one bank accounts.\\
\hline
Transaction & Any kind of money exchange associated with a bank account.\\
\hline
Transfer & A type of transaction that occurs between two parties.\\
\hline
Deposit & A type of transaction where the owner puts money in his own bank account.\\
\hline
Withdrawal & A type of transaction where the owner of the bank account removes money from his balance.\\
\hline
Database & A local or online container which holds data in an organized, efficient manner. \\
\hline
Server & a computer that is accessible on a network, on which a database and/or system may be hosted. The bank institutions' databases will be hosted on here. \\
\hline
Object-Oriented Programming & A programming paradigm which separates entities into objects, and uses the concept of inheritance of properties, polymorphism of objects, encapsulation of objects. We use this paradigm for its maintainability and structural benefits.\\
\hline
MVC - Model-View-Controller Architecture & An architectural pattern which strictly separates components into the model (manages the data and logic), the view (output of the model), and the controller (handling input and passing it to the model or view).\\
\hline
Interface & A component of a system by which other entities (be it humans or other systems) may engage in an exchange of data with the system in question. \\
\hline
API - Application Programming Interface & A protocol or set of functions which serve as a method of communication to a software system. It is a type of interface, and the one by which our system will communicate with the banking institutions' databases. \\
\hline
DAO - Data access object & An object that provides an abstract interface to some type of database or other persistence mechanism.\\
\hline

\end{tabular}}
\end{center}
\end{table}

\clearpage

\subsection{Subsystem X}

\subsubsection{Detailed Design Diagram}

UML class diagram depicting the internal structure of the subsystem,
accompanied by a paragraph of text describing the rationale of this design.

*Note: The above is a description of what to provide. Need to edit into our own




\subsubsection{Units Description}

List each class in this subsystem and write a short description of its purpose,
as well as notes or reminders useful for the programmers who will implement them.
List all attributes and functions of the class.

*Note: The above is a description of what to provide. Need to edit into our own

\section{Dynamic Design Scenarios} \label{Dynamic Models}

Describe some (at least two) important execution scenarios of the system using UML sequence diagrams.
These scenarios must demonstrate how the various subsystems and units are interacting to achieve a system-level service.
Units and subsystems depicted here must be compatible with the descriptions provided in
section \ref{sec:arch} and \ref{sec:detail}.

*Note: The above is a description of what to provide. Need to edit into our own


\subsection{Dynamic Models of System Interface}

We have chosen 3 major functionalities of the system (also known as use cases) in order to portray the interactions between the classes of the system. By using a sequence diagram, this will display the dynamics visually by showcasing the sequences of method calls when a particular use case begins functioning.

\clearpage

\subsubsection{Use Case 1: Create User Account}

The following scenario describes the actions that occur when the user clicks on the sign up button

\begin{figure}[H]
\includegraphics[width=\linewidth]{case1SequenceDiagram.png}
\caption{Use case 1 Sequence Diagram}
\label{fig:use-case-1-sequence-diagram}
\end{figure}

\clearpage

\subsubsection{Use Case 3: Add Bank Account to a User Account}

The following scenario describes the actions that occur when a user clicks the add button in the account list view.

\begin{figure}[H]
\includegraphics[width=\linewidth]{usecase3_sequence_diagram.png}
\caption{Use case 3 Sequence Diagram}
\label{fig:use-case-3-sequence-diagram}
\end{figure}

\clearpage

\subsubsection{Use Case 5: View Transactions for Specific Bank Account}

The following scenario describes the actions that occur when the user clicks the button<view transactions> for a specific bank account.

\begin{figure}[H]
\includegraphics[width=\linewidth]{usecase5.png}
\caption{Use case 5 Sequence Diagram}
\label{fig:use-case-5-sequence-diagram}
\end{figure}

\clearpage

\clearpage

\subsubsection{Use Case 6: View All Transactions from all Bank Accounts}

The following scenario describes the actions that occur when the user click the button "view all transactions" for viewing all transactions from all bank accounts.

\begin{figure}[H]
\includegraphics[width=\linewidth]{Use_Case_6_Sequence_Diagram.png}
\caption{UseCase 6 Sequence Diagram}
\label{fig:use-case-6-sequence-diagram}
\end{figure}

\clearpage

\section{Reference}

\begin{itemize}
\item User information: As our user and use-cases was based on feedback provided by our developers, our references lie mainly within our own team.
\item Craig Larman - Applying UML and Patterns
\item Greg Butler's course COMP 354 content
\item \href{http://web.mit.edu/ssit/cis/CISRequirements.html}{\textcolor{blue}{MIT Curricular Information System
Software Requirements Document}}
\item \href{https://resources.sei.cmu.edu/asset_files/TechnicalReport/2005_005_001_14621.pdf}{\textcolor{blue}{Carnegie Mellon Business Goals}}
\item \href{http://www.oracle.com/technetwork/testcontent/gettingstartedwithusecasemodeling-133857.pdf}{\textcolor{blue}{Use-Case: Oracle }}
\item \href{https://github.com/google/dagger}{\textcolor{blue}{Google Dagger Github}}

\end{itemize}

\end{document}
