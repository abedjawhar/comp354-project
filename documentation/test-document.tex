\documentclass[12pt]{article}

\usepackage{multirow}
\usepackage{tabularx}
\usepackage[table,xcdraw]{xcolor}

\usepackage{seqsplit}
\usepackage{url}

\usepackage{longtable}

\makeatletter
\g@addto@macro\UrlSpecials{\camelurl}
\def\camelurl{%
\count@`a
\loop
\mathcode\count@"8000
\uccode`\~\count@\uppercase{\edef~{\mathchar\the\count@\noexpand\breakifupper}}%
\ifnum\count@<`\z
\advance\count@\@ne
\repeat}

\def\breakifupper#1{%
\ifcat .\noexpand#1%
\ifnum`#1>40
\ifnum`#1<91
\penalty\z@
\fi\fi\fi
#1%
}

\makeatother

\pagestyle{empty}
\setcounter{secnumdepth}{2}

\topmargin=0cm
\oddsidemargin=0cm
\textheight=22.0cm
\textwidth=17cm
\parindent=0cm
\parskip=0.15cm
\topskip=0truecm
\raggedbottom
\abovedisplayskip=3mm
\belowdisplayskip=3mm
\abovedisplayshortskip=0mm
\belowdisplayshortskip=2mm
\normalbaselineskip=12pt
\normalbaselines

% environment slightly edited from https://tex.stackexchange.com/questions/10293/latex-template-for-use-cases
\newcommand\tabularhead[1]{
    \begin{table}[ht]
        \addtocounter{use case ID}{1}
        \caption{Use Case \arabic{use case ID} - #1}
        \vspace{0.2cm}
        \begin{tabular}{|p{0.2\linewidth}|p{0.70\linewidth}|}
            \hline
            \textbf{Action} & \textbf{#1} \\
            \hline}

        \newcommand\addrow[2]{#1 & #2\\ \hline}

            \newcommand\addmulrow[2]{ \begin{minipage}[t][][t]{2.5cm}#1\end{minipage}
                &\begin{minipage}[t][][t]{11cm}
                    \begin{enumerate}[itemsep=-1ex] #2   \end{enumerate}
                \end{minipage}\vfill\\ \hline}

            \newenvironment{usecase}{\tabularhead}
        {\hline\end{tabular}\end{table}}



        % cheaty non-functional requirement env

        \newcounter{req ID}
        \newcommand\tabularheadfsd[1]{
            \begin{table}[ht]
                \addtocounter{req ID}{1}
                \caption{Non-Functional Requirement \arabic{req ID} - #1}
                \vspace{0.2cm}
                \begin{tabular}{|p{0.2\linewidth}|p{0.70\linewidth}|}
                    \hline
                    \textbf{Action} & \textbf{#1} \\
                    \hline}

                \newenvironment{requirement}{\tabularheadfsd}
                {\hline\end{tabular}\end{table}}

                \begin{document}

                \vspace*{0.5in}
                \centerline{\bf\Large COMP 354}
                \centerline{\bf\Large Test Document for the project myMoney}

                \vspace*{0.5in}
                \centerline{\bf\Large Team PA-PK}

                \vspace*{0.5in}
                \centerline{\today}

                \vspace*{1.5in}
                \begin{table}[htbp]
                    \caption{Team}
                    \begin{center}
                        \begin{tabular}{|r | c|}
                            \hline
                            Name & ID Number \\
                            \hline\hline
                            Anne-Laure Ehresmann & 27858906 \\
                            \hline
                            Marc-Antoine Dube & 40029307 \\
                            \hline
                            Kadeem Caines & 26343600 \\
                            \hline
                            Abdel Rahman Jawhar & 27192142 \\
                            \hline
                            Keith Dion & 40036340 \\
                            \hline
                            Hrachya Hakobyan & 40041555 \\
                            \hline
                            Andrew-Smith & 40034936 \\
                            \hline
                            Dongyu Chen & 27241909 \\
                            \hline
                            Yauheni Karaniuk & 40005680 \\
                            \hline
                            Renny Xu & 40005262\\
                            \hline
                            Wei Wang & 40041116 \\
                            \hline
                        \end{tabular}
                    \end{center}
                \end{table}

                \begin{table}[htbp]
                    \caption{Revision history}
                    \begin{center}
                        \begin{tabular}{|r | c| c |}
                            \hline
                            Version & Date & Changes \\
                            \hline
                            1.0 & \nth{15} March 2018 & Completed test document \\
                            \hline
                        \end{tabular}
                    \end{center}
                \end{table}


                \tableofcontents
\listoffigures
\clearpage
\listoftables

\clearpage

\section{Introduction}

{\it
The introduction of the document provides an overview of the entire document,
briefly introducing what are its goals, and what information is to be found in it.
}

\section{Test Plan}

{\it
Describe what forms of testing you plan to do (unit, subsystem, integration),
describe briefly the schedule and resources for testing,
and
how you designed your test cases.

Indicate which qualities (from requirements) were tested and which qualities were not tested.
}

\subsection{System Level Test Cases}

{\it
All test cases for testing at the system level.
}

\subsubsection{Test Case 1} \label{tc:1}

\noindent
{\bf Purpose}\\
State the purpose of the test.
Indicate which requirement and which aspect of that requirement is being tested.

\noindent
{\bf Input Specification}\\
State the context for the test in terms of system state.
State the input test data. You may need to mention operations invoked as well as data for the operation.
You can cross-reference to actual file data specified in an appendix.

\noindent
{\bf Expected Output}\\
State the expected system response and output.
You can cross-reference to actual file data specified in an appendix.

\noindent
{\bf Traces to Use Cases}\\
State which requirements (at the level of use case and scenario) are tested by this test case.

\subsection{Subsystem Level Test Cases}

{\it
All test cases for testing at the subsystem level.
}

{\it
One subsection per subsystem
}

\subsubsection{Subsystem X}

\subsection{Unit Test cases}

\subsubsection{com.github.comp354project.model.user.UserService.createUser(User)}
\begin{longtable}{|m{4cm}|l|l|l|}
\caption[]{createUser(User)}
\hline
\cellcolor[HTML]{C0C0C0}\textbf{Tester Name} & \multicolumn{2}{p{13cm}|}{Hrachya}\\ \hline
\cellcolor[HTML]{C0C0C0}\textbf{Test Date} & \multicolumn{2}{p{13cm}|}{1/31/18}\\ \hline
\cellcolor[HTML]{C0C0C0}\textbf{Class Name} & \multicolumn{2}{p{13cm}|}{com.github.comp354project.model.user.UserService}\\ \hline
\cellcolor[HTML]{C0C0C0}\textbf{Method Name} & \multicolumn{2}{p{13cm}|}{createUser(User)}\\ \hline
\cellcolor[HTML]{C0C0C0}\textbf{Purpose} & \multicolumn{2}{p{13cm}|}{This test suite tests the creation of a user}\\ \hline
\cellcolor[HTML]{C0C0C0}\textbf{Use Cases} & \multicolumn{2}{p{13cm}|}{\begin{tabular}[c]{@{}l@{}}01\\ \end{tabular}}\\ \hline
\multicolumn{3}{|l|}{\cellcolor[HTML]{C0C0C0}\textbf{Test Scenarios}}\\ \hline
\multicolumn{3}{|l|}{createUser\_withNullUser\_shouldThrow}\\ \hline
\rowcolor[HTML]{C0C0C0} &  \multicolumn{2}{l|}{user} \\ \cline{2-3}
\multirow{-2}{*}{\cellcolor[HTML]{C0C0C0}\textbf{Input Specification}} & \multicolumn{2}{l|}{ null } \\ \hline
\cellcolor[HTML]{C0C0C0}\textbf{Expected Output} & \multicolumn{2}{p{13cm}|}{\begin{tabular}[c]{@{}l@{}}ValidationException is thrown\\ \end{tabular}}\\ \hline
\cellcolor[HTML]{C0C0C0}\textbf{Actual Output} & \multicolumn{2}{p{13cm}|}{\begin{tabular}[c]{@{}l@{}}ValidationException is thrown\\ \end{tabular}}\\ \hline
\cellcolor[HTML]{C0C0C0}\textbf{Bug Found} & \multicolumn{2}{p{13cm}|}{false}\\ \hline
\cellcolor[HTML]{C0C0C0}\textbf{Purpose} & \multicolumn{2}{p{13cm}|}{No null user can be saved in the database}\\ \hline
\multicolumn{3}{|l|}{testCreateUser\_withInvalidUser\_shouldThrow}\\ \hline
\rowcolor[HTML]{C0C0C0} &  user &  errors \\ \cline{2-3}
\multirow{-2}{*}{\cellcolor[HTML]{C0C0C0}\textbf{Input Specification}} & \begin{tabular}[c]{@{}l@{}}\end{tabular}& 4 \\ \hline
\cellcolor[HTML]{C0C0C0}\textbf{Expected Output} & \multicolumn{2}{p{13cm}|}{\begin{tabular}[c]{@{}l@{}}ValidationException is thrown\\ 4 exceptions are thrown because missing fields: username, password, firstname, lastname\\ \end{tabular}}\\ \hline
\cellcolor[HTML]{C0C0C0}\textbf{Actual Output} & \multicolumn{2}{p{13cm}|}{\begin{tabular}[c]{@{}l@{}}ValidationException is thrown\\ 4 exceptions are thrown because missing fields: username, password, firstname, lastname\\ \end{tabular}}\\ \hline
\cellcolor[HTML]{C0C0C0}\textbf{Bug Found} & \multicolumn{2}{p{13cm}|}{false}\\ \hline
\cellcolor[HTML]{C0C0C0}\textbf{Purpose} & \multicolumn{2}{p{13cm}|}{No empty value user can be saved in the database}\\ \hline
\multicolumn{3}{|l|}{testCreateUser\_withValidUser\_shouldReturnUser}\\ \hline
\rowcolor[HTML]{C0C0C0} &  \multicolumn{2}{l|}{user} \\ \cline{2-3}
\multirow{-2}{*}{\cellcolor[HTML]{C0C0C0}\textbf{Input Specification}} & \multicolumn{2}{l|}{ \begin{tabular}[c]{@{}l@{}}username: USERNAME\\ password: PASSWORD\\ firstName: FIRSTNAME\\ lastName: LASTNAME\\ \end{tabular}} \\ \hline
\cellcolor[HTML]{C0C0C0}\textbf{Expected Output} & \multicolumn{2}{p{13cm}|}{\begin{tabular}[c]{@{}l@{}}User ID was autogenerated upon save\\ The saved user is the same as the inputted user\\ \end{tabular}}\\ \hline
\cellcolor[HTML]{C0C0C0}\textbf{Actual Output} & \multicolumn{2}{p{13cm}|}{\begin{tabular}[c]{@{}l@{}}User ID was autogenerated upon save\\ The saved user is the same as the inputted user\\ \end{tabular}}\\ \hline
\cellcolor[HTML]{C0C0C0}\textbf{Bug Found} & \multicolumn{2}{p{13cm}|}{false}\\ \hline
\cellcolor[HTML]{C0C0C0}\textbf{Purpose} & \multicolumn{2}{p{13cm}|}{A valid user should be inserted in the database}\\ \hline
\multicolumn{3}{|l|}{testCreateUser\_withExistingUsername\_shouldThrow}\\ \hline
\rowcolor[HTML]{C0C0C0} &  \multicolumn{2}{l|}{user} \\ \cline{2-3}
\multirow{-2}{*}{\cellcolor[HTML]{C0C0C0}\textbf{Input Specification}} & \multicolumn{2}{l|}{ \begin{tabular}[c]{@{}l@{}}username: USERNAME\\ password: PASSWORD\\ firstName: FIRSTNAME\\ lastName: LASTNAME\\ \end{tabular}} \\ \hline
\cellcolor[HTML]{C0C0C0}\textbf{Expected Output} & \multicolumn{2}{p{13cm}|}{\begin{tabular}[c]{@{}l@{}}ValidationException is thrown\\ \end{tabular}}\\ \hline
\cellcolor[HTML]{C0C0C0}\textbf{Actual Output} & \multicolumn{2}{p{13cm}|}{\begin{tabular}[c]{@{}l@{}}ValidationException is thrown\\ \end{tabular}}\\ \hline
\cellcolor[HTML]{C0C0C0}\textbf{Bug Found} & \multicolumn{2}{p{13cm}|}{false}\\ \hline
\cellcolor[HTML]{C0C0C0}\textbf{Purpose} & \multicolumn{2}{p{13cm}|}{A user cannot be created if the username is already taken}\\ \hline
\end{longtable}

\subsubsection{com.github.comp354project.model.user.UserService.deleteBankAccount(Account)}
\begin{longtable}{|m{4cm}|l|l|l|l|}
\caption[]{deleteBankAccount(Account)}
\hline
\cellcolor[HTML]{C0C0C0}\textbf{Tester Name} & \multicolumn{3}{p{13cm}|}{Anne-Laure}\\ \hline
\cellcolor[HTML]{C0C0C0}\textbf{Test Date} & \multicolumn{3}{p{13cm}|}{3/7/18}\\ \hline
\cellcolor[HTML]{C0C0C0}\textbf{Class Name} & \multicolumn{3}{p{13cm}|}{com.github.comp354project.model.user.UserService}\\ \hline
\cellcolor[HTML]{C0C0C0}\textbf{Method Name} & \multicolumn{3}{p{13cm}|}{deleteBankAccount(Account)}\\ \hline
\cellcolor[HTML]{C0C0C0}\textbf{Purpose} & \multicolumn{3}{p{13cm}|}{This test suite tests the deletion of a bank account}\\ \hline
\cellcolor[HTML]{C0C0C0}\textbf{Use Cases} & \multicolumn{3}{p{13cm}|}{\begin{tabular}[c]{@{}l@{}}04\\ \end{tabular}}\\ \hline
\multicolumn{4}{|l|}{\cellcolor[HTML]{C0C0C0}\textbf{Test Scenarios}}\\ \hline
\multicolumn{4}{|l|}{testDeleteBankAccount\_withNullAccount\_ShouldThrow}\\ \hline
\rowcolor[HTML]{C0C0C0} &  \multicolumn{3}{l|}{account} \\ \cline{2-4}
\multirow{-2}{*}{\cellcolor[HTML]{C0C0C0}\textbf{Input Specification}} & \multicolumn{3}{l|}{ null } \\ \hline
\cellcolor[HTML]{C0C0C0}\textbf{Expected Output} & \multicolumn{3}{p{13cm}|}{\begin{tabular}[c]{@{}l@{}}ValidationException is thrown\\ \end{tabular}}\\ \hline
\cellcolor[HTML]{C0C0C0}\textbf{Actual Output} & \multicolumn{3}{p{13cm}|}{\begin{tabular}[c]{@{}l@{}}ValidationException is thrown\\ \end{tabular}}\\ \hline
\cellcolor[HTML]{C0C0C0}\textbf{Bug Found} & \multicolumn{3}{p{13cm}|}{false}\\ \hline
\cellcolor[HTML]{C0C0C0}\textbf{Purpose} & \multicolumn{3}{p{13cm}|}{No null account can be passed to the function}\\ \hline
\multicolumn{4}{|l|}{testDeleteBankAccount\_withoutBeingLoggedIn\_ShouldThrow}\\ \hline
\rowcolor[HTML]{C0C0C0} &  account &  \multicolumn{2}{l|}{testUser} \\ \cline{2-4}
\multirow{-2}{*}{\cellcolor[HTML]{C0C0C0}\textbf{Input Specification}} & \begin{tabular}[c]{@{}l@{}}ID: 1\\ user: testUser\\ bankName: TD\\ type: Checking\\ balance: 15823.12\\ \end{tabular}& \multicolumn{2}{l|}{ \begin{tabular}[c]{@{}l@{}}ID: 1\\ firstName: Hrachya\\ lastName: Hakobyan\\ username: admin\\ password: admin\\ email: sample@email.com\\ address: address\\ phone: 111111\\ \end{tabular}} \\ \hline
\cellcolor[HTML]{C0C0C0}\textbf{Expected Output} & \multicolumn{3}{p{13cm}|}{\begin{tabular}[c]{@{}l@{}}AuthenticationException is thrown\\ \end{tabular}}\\ \hline
\cellcolor[HTML]{C0C0C0}\textbf{Actual Output} & \multicolumn{3}{p{13cm}|}{\begin{tabular}[c]{@{}l@{}}AuthenticationException is thrown\\ \end{tabular}}\\ \hline
\cellcolor[HTML]{C0C0C0}\textbf{Bug Found} & \multicolumn{3}{p{13cm}|}{false}\\ \hline
\cellcolor[HTML]{C0C0C0}\textbf{Purpose} & \multicolumn{3}{p{13cm}|}{A user that is not authenticated cannot delete his accounts}\\ \hline
\multicolumn{4}{|l|}{testDeleteBankAccount\_withoutProperAuthorisation\_ShouldThrow}\\ \hline
\rowcolor[HTML]{C0C0C0} &  testUser &  user2 &  testAccount \\ \cline{2-4}
\multirow{-2}{*}{\cellcolor[HTML]{C0C0C0}\textbf{Input Specification}} & \begin{tabular}[c]{@{}l@{}}ID: 1\\ firstName: Hrachya\\ lastName: Hakobyan\\ username: admin\\ password: admin\\ email: sample@email.com\\ address: address\\ phone: 111111\\ \end{tabular}& \begin{tabular}[c]{@{}l@{}}username: username\\ password: password\\ firstName: firstname\\ lastName: lastname\\ ID: 999\\ \end{tabular}& \begin{tabular}[c]{@{}l@{}}ID: 1\\ user: testUser\\ bankName: TD\\ type: Checking\\ balance: 15823.12\\ \end{tabular}\\ \hline
\cellcolor[HTML]{C0C0C0}\textbf{Expected Output} & \multicolumn{3}{p{13cm}|}{\begin{tabular}[c]{@{}l@{}}AuthorisationException is thrown\\ \end{tabular}}\\ \hline
\cellcolor[HTML]{C0C0C0}\textbf{Actual Output} & \multicolumn{3}{p{13cm}|}{\begin{tabular}[c]{@{}l@{}}AuthorisationException is thrown\\ \end{tabular}}\\ \hline
\cellcolor[HTML]{C0C0C0}\textbf{Bug Found} & \multicolumn{3}{p{13cm}|}{false}\\ \hline
\cellcolor[HTML]{C0C0C0}\textbf{Purpose} & \multicolumn{3}{p{13cm}|}{A user cannot modify the accounts of another user}\\ \hline
\multicolumn{4}{|l|}{testDeleteBankAccount\_WithProperAuthorisation\_ShouldSucceed}\\ \hline
\rowcolor[HTML]{C0C0C0} &  testUser &  testAccount &  invocationCount \\ \cline{2-4}
\multirow{-2}{*}{\cellcolor[HTML]{C0C0C0}\textbf{Input Specification}} & \begin{tabular}[c]{@{}l@{}}ID: 1\\ firstName: Hrachya\\ lastName: Hakobyan\\ username: admin\\ password: admin\\ email: sample@email.com\\ address: address\\ phone: 111111\\ \end{tabular}& \begin{tabular}[c]{@{}l@{}}ID: 1\\ user: testUser\\ bankName: TD\\ type: Checking\\ balance: 15823.12\\ \end{tabular}& 1 \\ \hline
\cellcolor[HTML]{C0C0C0}\textbf{Expected Output} & \multicolumn{3}{p{13cm}|}{\begin{tabular}[c]{@{}l@{}}Execution of the deletion of the account once\\ \end{tabular}}\\ \hline
\cellcolor[HTML]{C0C0C0}\textbf{Actual Output} & \multicolumn{3}{p{13cm}|}{\begin{tabular}[c]{@{}l@{}}Execution of the deletion of the account once\\ \end{tabular}}\\ \hline
\cellcolor[HTML]{C0C0C0}\textbf{Bug Found} & \multicolumn{3}{p{13cm}|}{false}\\ \hline
\cellcolor[HTML]{C0C0C0}\textbf{Purpose} & \multicolumn{3}{p{13cm}|}{An authenticated user should succeed in deleting his own bank accounts}\\ \hline
\end{longtable}







{\it
All test cases for testing at the unit level.
}

{\it
One subsection per unit
}

\subsubsection{Unit X}

\section{Test Results}

{\it
List the tests, indicating which passed and which did not pass.
List requirements indicating the percentage of tests that passed for that requirement.
}

\section{References}

\appendix

\section{Description of Input Files}

Describe/include test data from input files.

\section{Description of Output Files}

Describe/include test expected output that are output files.

\end{document}